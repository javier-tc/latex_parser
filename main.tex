\documentclass[letterpaper,11pt]{article}
\input{preambuloejercitaciones}
\renewcommand{\title}{Ejemplo de ejercicios de Alternativas} %% AQUI PONER EL NOMBRE DE LA GUIA
\newcommand{\codigozipgrade}{HXXX}  %% AQUI PONER EL CODIGO CORRESPONDIENTE DE ZIPGRADE
\newcommand{\anchopregunta}{0.9\textwidth}
\newcommand{\anchosuficiencia}{0.8\textwidth}
\usepackage{wrapfig}
\input{preambuloejercitaciones}

\begin{document}

\begin{enumerate}
\begin{minipage}{\anchopregunta}
\item ``La antropóloga Constanza Tocornal, de la Universidad Católica Silva Henríquez, trabaja con ellos [los Selk’nam] en la reconstrucción de memorias orales e historias familiares, y en la revisión de fuentes archivísticas y documentales.\\
``El reconocimiento cultural y político del pueblo selk'nam tiene que considerar que el genocidio dificulta la continuidad cultural. En estas memorias familiares, hay procesos íntimos de invisibilización, miedo y violencia sufrida hacia su posibilidad de autoidentificarse como un pueblo, al que la sociedad le decía que estaba desaparecido. Eso también es parte de los componentes identitarios'', explica.\\
El proceso legal de reconocimiento no tiene que ver con pureza sanguínea, aclaran en la Corporación Selk’nam Chile. Los pueblos cambian y aunque hoy no habiten en el territorio ni hablen la lengua, mantienen ciertos rasgos culturales. Ellos mismos descubren parecidos cuando se reúnen. Hay también ciertas prácticas y habilidades en las familias, como el trabajo textil o en cuero que, "una vez que se reconoce la posibilidad del ancestro selk'nam y lo contrasta con relatos etnográficos, encuentra mayor explicación", agrega Tocornal.''
\begin{flushright}
\textit{Selk'nam: la reaparición de un pueblo que se creía extinguido}, Victoria Dannemann, DW (2021).
\end{flushright}
El fragmento anterior relata parte de la lucha del pueblo Selk´nam, alguna vez creído extinto, por lograr el reconocimiento de su existencia y vida en el presente. Considerando dicho contexto y lo planteado por la fuente, un mecanismo para la revalorización de la cultura nativa es:
\begin{flushleft}\begin{tabular}{@{\hspace{-.001\textwidth}}l@{\hspace{2pt}}p{.965\textwidth}}
A)& Tener un trauma histórico que genere miedo para guardar silencio y conservar la cultura local.\\
B)& Encontrar costumbres y tradiciones compartidas con otros para realizar investigaciones académicas.\\
C)& Ser sanguíneamente nativo con el fin de conformar una identidad biológica que sea más fácil de investigar.\\
D)& Realizar un estudio académico sobre la cultura de los indígenas como sujetos históricos objetivos.\\
E)& Permitir que cualquier persona aporte e influya en el conocimiento indígena con el fin de revivir la cultura nativa.\\ 
\end{tabular}\end{flushleft}%correcta
\begin{key} B
\end{key} 
\begin{hint}
\end{hint}
\begin{answer} Correcta B \\
\end{answer}
\begin{info} %Información técnica de la pregunta
\begin{flushleft}
Eje Temático: Formación Ciudadana\\
Contenido: Relaciones de conflicto y convivencia con los pueblos indígenas en Chile y el valor de la diversidad cultural. \\
Habilidad: Análisis de fuentes de investigación. \\
Dificultad: Muy fácil, fácil.\\
Capsula: Por determinar.\\
\end{flushleft} 
\end{info}
\end{minipage}\vfill$\;$ %1

\begin{minipage}{\anchopregunta}
\item ``Nosotros habitamos más de 11 mil años esta zona y queremos una participación activa de las decisiones que toma el Estado”, comenta a DF MAS Vladimir Reyes. “Hay una preocupación, porque nunca nos han consultado. Siempre llegan y hacen. Estamos preocupados por nuestro salar, vivimos en un desierto y el recurso hídrico es relevante para la supervivencia de nuestras comunidades”, agrega el dirigente.
Su objetivo es cambiar las cosas que están pasando actualmente. “Las empresas gravemente perjudicadas están afectando el medioambiente y usan el agua más de lo permitido. Hay lugares que se han ido secando, y eso es algo que no podemos permitir”, agrega Reyes.''
\begin{flushright}
\textit{“A más tardar el lunes”: el mensaje de los pueblos atacameños para concretar reunión con Boric sobre el litio}, Mateo Navas, DF MAS (2023)
\end{flushright}
Considerando lo planteado por el texto y las relaciones entre el Estado chileno y los pueblos originarios que enmarcan las transacciones del litio, es correcto concluir que:
\begin{flushleft}\begin{tabular}{@{\hspace{-.001\textwidth}}l@{\hspace{2pt}}p{.965\textwidth}}
A)& El encuentro representa la primera vez en la que el Estado llega a un acuerdo con las comunidades locales.\\
B)& Lo planteado por Vladimir Reyes es infundado, pues los atropellos del Estado  solo se han dado en el sur de Chile.\\
C)& El temor de los atacameños se encuentra justificado por años de abusos a las comunidades nativas.\\
D)& La explotación del litio y el lucro generado de esta es la principal preocupación de los atacameños.\\
E)& Las comunidades indígenas se encuentran completamente indiferentes a la extracción del litio.\\ 
\end{tabular}\end{flushleft}%correcta
\begin{key} C
\end{key} 
\begin{hint}
\end{hint}
\begin{answer} Correcta C \\
\end{answer}
\begin{info} %Información técnica de la pregunta
\begin{flushleft}
Eje Temático: Formación Ciudadana.\\
Contenido: Relaciones de conflicto y convivencia con los pueblos indígenas en Chile y el valor de la diversidad cultural.\\
Habilidad: Pensamiento crítico.\\
Dificultad: Fácil, media.\\
Capsula: Por determinar.\\
\end{flushleft} 
\end{info}
\end{minipage}\vfill$\;$ %2

\begin{minipage}{\anchopregunta}
\item Observe la siguiente imagen:
\begin{center}
\includegraphics[height=8cm]{imagenes/imagen1.png}
\end{center}
\begin{flushright}
\textit Fuente: ONG Desierto Vestido.
\end{flushright}
Desierto Vestido es una ONG radicada en Tarapacá que se encarga de generar conciencia y combatir la industria del Fast Fashion, pues las empresas textiles que la realizan producen ropa muy barata, pero extremadamente contaminante, de baja vida útil y que termina en grandes vertederos como los del norte de Chile. Considerando el contexto anterior y la idea de desarrollo sustentable, ¿qué acciones puede tomar el Estado para mitigar el efecto de dicha contaminación a corto plazo?
\begin{flushleft}\begin{tabular}{@{\hspace{-.001\textwidth}}l@{\hspace{2pt}}p{.965\textwidth}}
A)& Generar campañas de conciencia sobre el impacto medioambiental del Fast Fashion.\\
B)& Competir contra las grandes marcas de ropa con vestimenta artesanal y/o reciclada.\\
C)& Multar a las empresas con más emisiones o que usen grandes vertederos de ropa.\\
D)& Investigar nuevas formas de fabricación de ropa con el fin de abaratar costos.\\
E)& Expandir el uso de Fast Fashion para hacerle competencia y bajar sus ganancias.\\ 
\end{tabular}\end{flushleft}%correcta
\begin{key} C
\end{key} 
\begin{hint}
\end{hint}
\begin{answer} Correcta C \\
\end{answer}
\begin{info} %Información técnica de la pregunta
\begin{flushleft}
Eje Temático: Formación Ciudadana.\\
Contenido: El impacto del proceso de industrialización en el medioambiente y su relación con el debate actual sobre el desarrollo sostenible.\\
Habilidad: Pensamiento crítico.\\
Dificultad: Muy fácil, fácil. \\
Capsula: Por determinar \\
\end{flushleft} 
\end{info}
\end{minipage}\vfill$\;$ %3

\begin{minipage}{\anchopregunta}
\item Lee el siguiente poema:
\begin{center}
\begin{tabular}{ll}
\begin{tabular}[c]{@{}l@{}}Comienza mi jornada cuando sale el sol\\ Tengo 12 años, vivo en la desolación\\ Acá, en otra dimensión\\ Mis pequeñas manos son la producción\\ Miles de juguetes con los que podrán jugar\\ Allá, niños como yo \end{tabular} & \begin{tabular}[c]{@{}l@{}} Víctimas reales de un juego demencial\\ La economía del mercado\\ Busca qué es más fácil de explotar\\ La macroproducción que nos ofrece bienestar\\ Son millones de niños esclavos\\ Son niños esclavos, condenados.\end{tabular}
\end{tabular}
\end{center}
\begin{flushright}
\textit{Los hijos bastardos de la globalización}, Ska-P, 2008.
\end{flushright}
Ska-P es una banda de Ska de ideología anarquista que es bastante conocida en Europa. Considerando lo planteado por la fuente, ¿qué aspecto negativo de la globalización es destacado por el grupo?
\begin{flushleft}\begin{tabular}{@{\hspace{-.001\textwidth}}l@{\hspace{2pt}}p{.965\textwidth}}
A)& Las precarias condiciones laborales que pueden darse en España.\\
B)& La excesiva contaminación producida por el capitalismo desenfrenado.\\
C)& La desigualdad entre sectores poblacionales por el modelo productivo.\\
D)& El abandono de miles de niños en África debido a las guerras regionales.\\
\end{tabular}\end{flushleft}%correcta
\begin{key} C
\end{key} 
\begin{hint}
\end{hint}
\begin{answer} Correcta C \\
\end{answer}
\begin{info} %Información técnica de la pregunta
\begin{flushleft}
Eje Temático: Formación Ciudadana.\\
Contenido: El impacto del proceso de industrialización en el medioambiente y su relación con el debate actual sobre el desarrollo sostenible.\\
Habilidad: Análisis de fuentes.\\
Dificultad: Fácil, media. \\
Capsula: Por determinar \\
\end{flushleft} 
\end{info}
\end{minipage}\vfill$\;$ %61

\begin{minipage}{\anchopregunta}
\item Observa el siguiente gráfico
\begin{center}
    \includegraphics[height=8cm]{imagenes/imagen2.png}
\end{center}
La materia particulada (PM 2.5) son partículas contaminantes que, en su mayoría, provienen de la combustión. Considerando los datos entregados por el gráfico, ¿cuál de las siguientes conclusiones es correcta?:
\begin{flushleft}\begin{tabular}{@{\hspace{-.001\textwidth}}l@{\hspace{2pt}}p{.965\textwidth}}
A)& La producción humana está directamente relacionada con el material particulado.\\
B)& El descenso de PM 2.5 en pandemia se debe exclusivamente a las muertes por COVID.\\
C)& La quema de madera es la principal causa de las PM 2.5 en las grandes ciudades.\\
D)& El usar energías renovables solucionaría el problema de forma rápida y barata.\\
E)& Las altas tasas de PM 2.5 en Inglaterra se deben a su historia como país industrial.\\ 
\end{tabular}\end{flushleft}%correcta
\begin{key} A
\end{key} 
\begin{hint}
\end{hint}
\begin{answer} Correcta A \\
\end{answer}
\begin{info} %Información técnica de la pregunta
\begin{flushleft}
Eje Temático: Formación Ciudadana.\\
Contenido: El impacto del proceso de industrialización en el medioambiente y su relación con el debate actual sobre el desarrollo sostenible.\\
Habilidad: Análisis de fuentes de investigación.\\
Dificultad: Media.\\
Capsula: Por determinar \\
\end{flushleft} 
\end{info}
\end{minipage}\vfill$\;$ %4

\begin{minipage}{\anchopregunta}
\item ``En este orden de ideas, el voto en blanco importaría un “voto de protesta”, distinto respecto de otras conductas electorales no convencionales, como la abstención o la anulación del voto. Quien emite un voto en blanco no se excluye a sí mismo de un proceso electoral en particular ni del sistema político en general, sino que se erige en partícipe de uno y otro. Desde ese punto puede distinguirse claramente de la abstención. Por otro lado, quien emite un voto en blanco no expresa preferencia electoral por ninguno de los candidatos o partidos en competencia. En cambio, quien emite un voto nulo expresa una preferencia electoral, pero lo hace incorrecta o ilegítimamente en cuanto no identifica dicha preferencia de manera singular o anónima. Así, se afirma que el voto en blanco “es un voto que se emite desde la no preferencia (y hasta desde el rechazo) por todas las opciones electorales concurrentes, por todas las candidaturas\\De aceptarse estas ideas podría incluso decirse que el voto en blanco tiene un valor expresivo más importante que el abstencionismo, pues se adhiere al proceso democrático.''
\begin{flushright}
\textit{Validez del voto en blanco en Chile: un análisis crítico},Viviana Ponce de León, Revista de derecho (2019).
\end{flushright}
Al considerar el concepto de participación ciudadana en un marco electoral, podemos desprender que para la autora:
\begin{flushleft}\begin{tabular}{@{\hspace{-.001\textwidth}}l@{\hspace{2pt}}p{.965\textwidth}}
A)& Considera al voto blanco, junto a la abstención y el voto nulo, como expresiones que carecen de participación ciudadana.\\
B)& Entiende el voto blanco como algo más importante que los votos tradicionales por mostrar realmente el sentir ciudadano.\\
C)& Comprende que el voto nulo solamente expresa la indecisión de un votante al momento de elegir a dos políticos similares.\\
D)& Abraza la idea de la abstinencia como forma de participación ciudadana, pues esta crítica la validez de todo el proceso electoral.\\
E)& Concluye que el voto blanco es una forma de la participación ciudadana por su rechazo a los candidatos, pero no al proceso.\\ 
\end{tabular}\end{flushleft}%correcta
\begin{key} E
\end{key} 
\begin{hint}
\end{hint}
\begin{answer} Correcta E \\
\end{answer}
\begin{info} %Información técnica de la pregunta
\begin{flushleft}
Eje Temático: Formación Ciudadana.\\
Contenido: La democracia considerando sus fundamentos, atributos y dimensiones.\\
Habilidad: Análisis de fuentes de investigación \\
Dificultad: Fácil, media\\
Capsula: Por determinar \\
\end{flushleft} 
\end{info}
\end{minipage}\vfill$\;$ %5

\begin{minipage}{\anchopregunta}
\item Observe el siguiente meme:
\begin{center}
    \includegraphics[height=8cm]{imagenes/imagen3.png}
\end{center}
Si se compara el meme presente con la idea de ciudadanía activa, es posible llegar a la siguiente conclusión:
\begin{flushleft}\begin{tabular}{@{\hspace{-.001\textwidth}}l@{\hspace{2pt}}p{.965\textwidth}}
A)& El meme demuestra ciudadanía activa por el hecho de ir a votar y tener a un candidato en mente.\\
B)& El sufragar es algo innecesario cuando se es un ciudadano activo por su poco impacto en el país.\\
C)& El votar, para un ciudadano activo, es solamente necesario con las elecciones presidenciales.\\
D)& Los ciudadanos pasivos se quedan en su hogar siempre, obviando su derecho a sufragar por su desconexión.\\
E)& El meme muestra a un ciudadano pasivo, pues su participación en política se reduce al acto de votar.\\ 
\end{tabular}\end{flushleft}%correcta
\begin{key} E
\end{key} 
\begin{hint}
\end{hint}
\begin{answer} Correcta E \\
\end{answer}
\begin{info} %Información técnica de la pregunta
\begin{flushleft}
Eje Temático: Formación Ciudadana\\
Contenido: La democracia considerando sus fundamentos, atributos y dimensiones.\\
Habilidad: Pensamiento crítico.\\
Dificultad: Media, difícil.\\
Capsula: Por determinar \\
\end{flushleft} 
\end{info}
\end{minipage}\vfill$\;$ %6

\begin{minipage}{\anchopregunta}
\item ``Jaraquemada agregó que los diputados y las diputadas tienen todo el derecho y las facultades para pedir transparencia, aunque, en rigor, ellos igual fiscalizan los actos de gobierno y el Servel no es del Gobierno. Pero tienen todos los derechos, las facultades para pedir la transparencia y procurar que, en el futuro, se mantenga esta fe de porque estos procesos son tan transparentes y tan confiables. Pero es distinto a instalar esta percepción de fraude, que ya hemos visto lo que ha pasado en Estados Unidos; en México, que es una estrategia que ha utilizado el presidente actual de atacar al sistema electoral; y lo mismo que está ocurriendo en Brasil. Vimos las consecuencias que tuvo en Estados Unidos, con el asalto al Capitolio. Por lo tanto, creo que eso debería ser totalmente criticable y repudiado, por así decirlo, por todas las personas que creemos que hay que cuidar las instituciones y, especialmente, un Servicio Electoral que nunca nos ha dado alguna sospecha, ni la más mínima, de que, efectivamente, los procesos no se lleven bien''
\begin{flushright}
\textit{Chile Transparente y las acusaciones de fraude contra el Servel: "Son ataques infundados"} Pablo Aravena.
\end{flushright}
La declaración de María Jaraquemada, directora de la fundación Chile Transparente, se enmarca dentro de los dichos de un diputado, que acusaba al Servel de corrupción en el plebiscito de 2022. Considerando lo planteado en el texto y el contexto que lo acompaña, podemos concluir que:
\begin{flushleft}\begin{tabular}{@{\hspace{-.001\textwidth}}l@{\hspace{2pt}}p{.965\textwidth}}
A)& El SERVEl se muestra reticente a transparentar sus movimientos debido al fraude electoral.\\
B)& La absoluta desconfianza en algunas instituciones es riesgoso para la democracia.\\
C)& En las últimas décadas el fraude electoral ha sido común en Chile y el mundo.\\
D)& Las críticas a la institucionalidad solo provienen de la ciudadanía al no ser parte de ella.\\
E)& El SERVEL es un organismo que favorece al gobierno de turno al depender de él.\\ 
\end{tabular}\end{flushleft}%correcta
\begin{key} B
\end{key} 
\begin{hint}
\end{hint}
\begin{answer} Correcta B \\
\end{answer}
\begin{info} %Información técnica de la pregunta
\begin{flushleft}
Eje Temático: Formación Ciudadana.\\
Contenido: Características y funcionamiento de la institucionalidad democrática en Chile.\\
Habilidad: Análisis de fuentes de investigación.\\
Dificultad: Por determinar\\
Capsula: Por determinar \\
\end{flushleft} 
\end{info}
\end{minipage}\vfill$\;$ %7

\begin{minipage}{\anchopregunta}
\item Organismo autónomo, surgido en la Reforma Procesal Penal. Su misión es dirigir a las policías de investigación, proteger a las víctimas y testigos y, de ser necesario, presentar a los imputados ante tribunales. ¿A qué organismo corresponde esta definición?
\begin{flushleft}\begin{tabular}{@{\hspace{-.001\textwidth}}l@{\hspace{2pt}}p{.965\textwidth}}
A)& Ministerio Público.\\
B)& Tribunal Constitucional.\\
C)& Policía de Investigaciones.\\
D)& Defensoría Penal Pública\\
E)& Banco Central.\\ 
\end{tabular}\end{flushleft}%correcta
\begin{key} A
\end{key} 
\begin{hint}
\end{hint}
\begin{answer} Correcta A \\
\end{answer}
\begin{info} %Información técnica de la pregunta
\begin{flushleft}
Eje Temático: Formación Ciudadana.\\
Contenido: Características y funcionamiento de la institucionalidad democrática en Chile.\\
Habilidad: Análisis de fuentes de investigación.\\
Dificultad: Muy fácil\\
Capsula: Por determinar \\
\end{flushleft} 
\end{info}
\end{minipage}\vfill$\;$ %62

\begin{minipage}{\anchopregunta}
\item Observa a siguiente imagen:
\begin{center}
    \includegraphics[height=10cm]{imagenes/imagen4.png}
\end{center}
La principal crítica a la justicia chilena que se ve retratada es:
\begin{flushleft}\begin{tabular}{@{\hspace{-.001\textwidth}}l@{\hspace{2pt}}p{.965\textwidth}}
A)& La carencia de cárceles que ofrezcan reinserción y rehabilitación a los criminales.\\
B)& El control de los aparatos de justicia por gente incompetente y poco calificada.\\
C)& La gran lentitud de la Fiscalía en realizar investigaciones que sean provechosas.\\
D)& La corrupción y el juego de influencias que alteran el justo proceder de las sentencias.\\
E)& La saturación del sistema con  burócratas que se encuentran en constante paro.\\ 
\end{tabular}\end{flushleft}%correcta
\begin{key} D
\end{key} 
\begin{hint}
\end{hint}
\begin{answer} Correcta D \\
\end{answer}
\begin{info} %Información técnica de la pregunta
\begin{flushleft}
Eje Temático: Formación Ciudadana.\\
Contenido: Características y funcionamiento de la institucionalidad democrática en Chile.\\
Habilidad: Análisis de fuentes de investigación.\\
Dificultad: Fácil, Media.\\
Capsula: Por determinar \\
\end{flushleft} 
\end{info}
\end{minipage}\vfill$\;$ %8

\begin{minipage}{\anchopregunta}
\item Chile es una República, por lo que los poderes del Estado se encuentran divididos y delimitados en la Constitución, pero eso no implica que se encuentren totalmente aislados los unos de los otros. Según la Constitución, ¿En qué instancia donde los poderes del Estado actúan sobre la potestad tradicional de otro poder?
\begin{flushleft}\begin{tabular}{@{\hspace{-.001\textwidth}}l@{\hspace{2pt}}p{.965\textwidth}}
A)& Mediante la participación del poder judicial en la legislación.\\
B)& Cuando el TRICEL, representante del judicial, califica elecciones.\\
C)& Con la elección, por parte absoluta del presidente, del Tribunal Constitucional.\\
D)& Con los debates parlamentarios sobre el actuar del gobierno.\\
E)& Mediante la función colegisladora del Presidente de la República.\\ 
\end{tabular}\end{flushleft}%correcta
\begin{key} E
\end{key} 
\begin{hint}
\end{hint}
\begin{answer} Correcta E \\
\end{answer}
\begin{info} %Información técnica de la pregunta
\begin{flushleft}
Eje Temático: Formación Ciudadana.\\
Contenido: Características y funcionamiento de la institucionalidad democrática en Chile.\\
Habilidad: Análisis de fuentes de investigación.\\
Dificultad: Media, difícil.\\
Capsula: Por determinar \\
\end{flushleft} 
\end{info}
\end{minipage}\vfill$\;$ %63

\begin{minipage}{\anchopregunta}
\item Observa la siguiente tabla:\\
\begin{center}
\begin{tabular}{|c|c|l|l|l|l|}
\hline
\multicolumn{1}{|l|}{N°} & \multicolumn{1}{l|}{Nombre} & Edad & Género    & Nacionalidad      & Otra información                \\ \hline
1                        & Mariano                     & 26   & Masculino & Argentina/chilena & Acusado de robar garbanzos.      \\ \hline
2                        & Sofia                       & 22   & Femenino  & Colombiana        & En Chile desde 2010              \\ \hline
3                        & Nicolás                     & 20   & Masculino & Chilena           & Está en el extranjero desde 2015 \\ \hline
\end{tabular}
\end{center}\\
De acuerdo a la información entregada y sus conocimientos previos, ¿Qué personas se encuentran habilitadas para votar en una elección de alcalde?
\begin{flushleft}\begin{tabular}{@{\hspace{-.001\textwidth}}l@{\hspace{2pt}}p{.965\textwidth}}
A)& Mariano.\\
B)& Nicolás.\\
C)& Mariano y Sofía.\\
D)& Sofía y Nicolás.\\
E)& Todos pueden votar.\\ 
\end{tabular}\end{flushleft}%correcta
\begin{key} C
\end{key} 
\begin{hint}
\end{hint}
\begin{answer} Correcta C \\
\end{answer}
\begin{info} %Información técnica de la pregunta
\begin{flushleft}
Eje Temático: Formación Ciudadana\\
Contenido: Características y funcionamiento de la institucionalidad democrática en Chile.\\
Habilidad: Análisis de fuentes de investigación.\\
Dificultad: Difícil, muy difícil.\\
Capsula: Por determinar \\
\end{flushleft} 
\end{info}
\end{minipage}\vfill$\;$ %9

\begin{minipage}{\anchopregunta}
\item ``Polarizados y divididos, los estadounidenses se fijan cada vez más en el papel de las redes sociales y de los principales medios de comunicación y en cómo los hechos son debatidos y distorsionados: "La gente recibe versiones cada vez más extremas de lo que ya ha consumido", debido a los algoritmos de las redes sociales, dijo Regina Lawrence, decana asociada de la Facultad de Periodismo y Comunicación de la Universidad de Oregón.\\En lo que respecta al corrosivo diálogo de los medios de comunicación, Lawrence dijo que deseaba que las instituciones políticas abordaran de forma más decidida los acontecimientos del 6 de enero, para que los medios de comunicación lo hicieran de esta misma manera.\\Más allá de los medios, hay formas más sencillas de comunicación que pueden ayudar a mejorar el consenso público sobre los acontecimientos ocurridos el 6 de enero, y sobre a quiénes hay que responsabilizar: "Hay buenas investigaciones que sugieren que, por muy difícil que sea, se debe escuchar lo más posible a aquellas personas que tienen esas creencias más extremas; entender que las razones por las que creen lo que creen puede resultar realmente útil", dijo Lawrence.''
\begin{flushright}
\textit{Cómo el asalto al Capitolio cambió -o no- a Estados Unidos}, John Marshall, DW (2022). 
\end{flushright}
Considerando lo planteado en el fragmento anterior y sus conocimientos previos, ¿qué pueden hacer los ciudadanos para combatir las fake news?:
\begin{flushleft}\begin{tabular}{@{\hspace{-.001\textwidth}}l@{\hspace{2pt}}p{.965\textwidth}}
A)& Concientizar a otros de la existencia de medios de información variados y verificados.\\
B)& Renunciar al uso de redes sociales para la obtención y difusión de información.\\
C)& Asumir como verdaderas las versiones que entregan las organizaciones estatales.\\
D)& Creer en lo que ellos mismos puedan ver y en la información anecdótica de terceros.\\
E)& Votar para que el Estado tenga un control directo sobre los medios de comunicación.\\ 
\end{tabular}\end{flushleft}%correcta
\begin{key} A
\end{key} 
\begin{hint}
\end{hint}
\begin{answer} Correcta A \\
\end{answer}
\begin{info} %Información técnica de la pregunta
\begin{flushleft}
Eje Temático: Educación Ciudadana.\\
Contenido: Oportunidades y riesgos de los medios de comunicación masiva y del uso las nuevas tecnologías de la información en el marco de una sociedad democrática.\\
Habilidad: Pensamiento crítico.\\
Dificultad: Media, difícil\\
Capsula: Por determinar \\
\end{flushleft} 
\end{info}
\end{minipage}\vfill$\;$ %10

\begin{minipage}{\anchopregunta}
\item La desinformación es uno de los principales problemas a los que se enfrentan las democracias actuales, pero no por ello es imposible de superar. ¿Cuál es una medida que puede tomar una persona para asegurarse que las noticias que lea sean verídicas?
\begin{flushleft}\begin{tabular}{@{\hspace{-.001\textwidth}}l@{\hspace{2pt}}p{.965\textwidth}}
A)& Revisar si estas fueron compartidas por figuras importantes.\\
B)& Analizar la calidad visual del medio que produjo la noticia.\\
C)& Comparar la noticia con otras fuentes de información sobre el tema.\\
D)& Asegurarse de que la noticia tenga fuentes que respalden lo dicho.\\
E)& Comprobar que la noticia venga de un medio de comunicación tradicional.\\ 
\end{tabular}\end{flushleft}%correcta
\begin{key} C
\end{key}
\begin{hint}
\end{hint}
\begin{answer} Correcta C \\
\end{answer}
\begin{info} %Información técnica de la pregunta
\begin{flushleft}
Eje Temático: Educación Ciudadana.\\
Contenido: Oportunidades y riesgos de los medios de comunicación masiva y del uso las nuevas tecnologías de la información en el marco de una sociedad democrática.\\
Habilidad: Pensamiento crítico.\\
Dificultad: Media, difícil.\\
Capsula: Por determinar \\
\end{flushleft} 
\end{info}
\end{minipage}\vfill$\;$ %11
%%%%%%%%%%%%%%%%%%%%%%INICIA SECCIÓN HISTORIA%%%%%%%%%%%%%%%%%%%%%%
\begin{minipage}{\anchopregunta}
\item Observa los siguientes cuadros:
\begin{center}
    \includegraphics[height=8cm]{imagenes/imagen5.jpg}
\end{center}
Teniendo en cuenta que los cuadros retratan eventos pertenecientes a procesos más grandes, y usando sus conocimientos previos, es posible concluir que la relación entre ambos procesos radica en que:
\begin{flushleft}\begin{tabular}{@{\hspace{-.001\textwidth}}l@{\hspace{2pt}}p{.965\textwidth}}
A)& El terror ocasionado por la monarquía francesa en Haití será imitado durante la Reconquista.\\
B)& Las independencias americanas fueron causadas directamente por el fin de la monarquía francesa. \\
C)& La toma de la Bastilla en Perú es posible mediante la ayuda del Ejército Libertador del Perú.\\
D)& Los revolucionarios franceses demostraron a los americanos que las ideas ilustradas eran viables. \\
E)& Los conflictos en Europa le mostraron a las colonias que podían acabar con el dominio imperial. \\ 
\end{tabular}\end{flushleft}%correcta
\begin{key} D
\end{key} 
\begin{hint}
\end{hint}
\begin{answer} Correcta D \\
\end{answer}
\begin{info} %Información técnica de la pregunta
\begin{flushleft}
Eje Temático: Historia: Mundo, América y Chile.\\
Contenido: Las ideas republicanas y liberales y su relación con las transformaciones
ocurridas en América y Europa durante el siglo XIX.\\
Habilidad: Análisis de fuentes de investigación\\
Dificultad: Media.\\
Capsula: Por determinar \\
\end{flushleft} 
\end{info}
\end{minipage}\vfill$\;$ %12

\begin{minipage}{\anchopregunta}
\item El liberalismo fue un movimiento político, cultural y económico que tuvo sus inicios a finales del siglo XVIII y principios del siglo XIX. Eran continuadores de la Ilustración y su influencia se extendió a lo largo del mundo. Dentro de sus principales ideas podemos encontrar:
\begin{flushleft}\begin{tabular}{@{\hspace{-.001\textwidth}}l@{\hspace{2pt}}p{.965\textwidth}}
A)& La unión de los ciudadanos en torno a la figura nacionalista del rey.\\
B)& El apoyo unánime al voto universal y a los derechos de las mujeres.\\
C)& La creación de una sociedad laica producto del ateísmo de los liberales.\\
D)& El libre intercambio de bienes con un gran énfasis la propiedad privada.\\
E)& El progreso indefinido donde el trabajo es considerado como algo secundario.\\ 
\end{tabular}\end{flushleft}%correcta
\begin{key} D
\end{key} 
\begin{hint}
\end{hint}
\begin{answer} Correcta D \\
\end{answer}
\begin{info} %Información técnica de la pregunta
\begin{flushleft}
Eje Temático: Historia: Mundo, América y Chile.\\
Contenido: Las ideas republicanas y liberales y su relación con las transformaciones ocurridas en América y Europa durante el siglo XIX.\\
Habilidad: Pensamiento temporal y espacial.\\
Dificultad: Muy fácil, fácil.\\
Capsula: Por determinar \\
\end{flushleft} 
\end{info}
\end{minipage}\vfill$\;$ %13

\begin{minipage}{\anchopregunta}
\item ``La ilimitada soberanía de las dinastías, de los nobles, de las ciudades y villas imperiales fue una adquisición revolucionaria a expensas de la nación y de su unidad. Me ha parecido siempre algo monstruoso el hecho de que la frontera que separa al habitante sajón de Salzwedel, del sajón de Brunswick, cerca de Lûchow, frontera difícil de reconocer a causa de sus pantanos y páramos, obligue a aquellos dos sajones a pertenecer a dos diferentes entidades nacionales, quizá enemiga la una de la otra, de las cuales una fue regida desde Berlín, la otra desde Londres, y más tarde desde Hannover.''
\begin{flushright}
\textit{Pensamientos y recuerdos} Otto Von Bismarck
\end{flushright}
Del texto anterior y, usando sus conocimientos previos, podemos inferir que:
\begin{flushleft}\begin{tabular}{@{\hspace{-.001\textwidth}}l@{\hspace{2pt}}p{.965\textwidth}}
A)& El liberal Bismarck instauró un sistema afín a sus ideas  tras la unificación alemana.\\
B)& Ya que la nación alemana es antigua, la unificación de 1871 es una vuelta al pasado.\\
C)& Todos los pueblos germanos se aliaron para alcanzar la unificación de Alemania.\\
D)& La idea de nación era antítesis a la monarquía instaurada en Alemania luego de 1871.\\
E)& Para Bismarck ni la geografía ni el dominio externo debían separar a una nación.\\ 
\end{tabular}\end{flushleft}%correcta
\begin{key} E
\end{key} 
\begin{hint}
\end{hint}
\begin{answer} Correcta E \\
\end{answer}
\begin{info} %Información técnica de la pregunta
\begin{flushleft}
Eje Temático: Historia: Mundo, América y Chile.\\
Contenido: Los impactos del surgimiento del Estado nación en América y Europa en el siglo XIX.\\
Habilidad: Análisis de fuentes de investigación.\\
Dificultad: Difícil, muy difícil.\\
Capsula: Por determinar \\
\end{flushleft} 
\end{info}
\end{minipage}\vfill$\;$ %14

\begin{minipage}{\anchopregunta}
\item ``Simón Bolívar creía necesario restaurar las instituciones liberales con la participación de los ciudadanos como únicos detentores de la soberanía. Este principio republicano fue una constante en los planteamientos de Bolívar, siempre y cuando esa soberanía no fuera contradictoria con su definición de libertad; de lo contrario, el ejercicio de la fuerza podía ser aceptado y reemplazar por un tiempo determinado la soberanía del pueblo.''\\
\begin{flushright}
\textit{Elecciones en la Gran Colombia, 1818-1830}, Nohra Palacios, 2021.
\end{flushright}
La Gran Colombia fue un país americano que surgió en los albores de la Independencia y, al igual que el resto de nacientes países, se vio inspirada por los principios liberales. Usando la fuente, es posible establecer la siguiente similitud con el proceso de conformación nacional chileno: \\
\begin{flushleft}\begin{tabular}{@{\hspace{-.001\textwidth}}l@{\hspace{2pt}}p{.965\textwidth}}
A)& La importancia del voto popular al formar democracias representativas donde todos pueden votar.\\
B)& La existencia de disenso político entre la clase gobernante y el uso de violencia para resolverlo.\\
C)& La búsqueda de un gran país americano que sea capaz de rivalizar con las potencias.\\
D)& La creación de un ideal republicano que busque dejar atrás el modelo económico colonial.\\
E)& Las diferencias políticas con los sectores populares, quienes seguían siendo fervientes realistas.\\ 
\end{tabular}\end{flushleft}%correcta
\begin{key} B
\end{key} 
\begin{hint}
\end{hint}
\begin{answer} Correcta B \\
\end{answer}
\begin{info} %Información técnica de la pregunta
\begin{flushleft}
Eje Temático: Historia: Mundo, América y Chile.\\
Contenido: Las ideas republicanas y liberales y su relación con las transformaciones ocurridas en América y Europa durante el siglo XIX.\\
Habilidad: Pensamiento temporal y espacial.\\
Dificultad: Media, difícil.\\
Capsula: Por determinar \\
\end{flushleft} 
\end{info}
\end{minipage}\vfill$\;$ %15

\begin{minipage}{\anchopregunta}
\item ``El final del siglo XIX y el comienzo del siglo XX expresaron la apoteosis de la idea de Progreso. Y esa apoteosis se materializó en un deslumbrante objeto creado por el homo faber. Un grande y maravilloso objeto. Un barco. Se le puso un nombre grandioso y se lo lanzó a las aguas. Se lo llamó Titanic. Era el mes de abril de 1912. En el Titanic cabían todas las clases. Estaban los que habían triunfado en la sociedad de competencia y los que no habían triunfado pero también tenían un lugar en el enorme barco; iban en tercera, pero iban. No estaban excluidos de la gran marcha hacia el Progreso.

El capitalismo era así: como ese gran barco. No eliminaba las desigualdades (jamás se lo había propuesto) pero construía un enorme espacio en el que todos entraban. ¿Para qué pedir más? ¿No era suficiente con mantener la humanidad a flote? Y si algunos pedían más, sólo necesitaban esperar. La tecnología era tan poderosa, avanzaba tan velozmente que acabaría por dar, si no igualdad, felicidad para todo el mundo.''\\
\begin{flushright}
\textit{Titanic},  Pablo Feinmann, 1998.
\end{flushright}
Considerando lo planteado por el texto y el ideal de progreso europeo durante el siglo XIX, es correcto afirmar que:
\begin{flushleft}\begin{tabular}{@{\hspace{-.001\textwidth}}l@{\hspace{2pt}}p{.965\textwidth}}
A)& Los burgueses creían en el trabajo como medio para alcanzar el progreso individual.\\
B)& La idea de progreso indefinido era exclusiva del pensamiento capitalista liberal.\\
C)& Los avances tecnológicos del XIX ayudaron a disminuir la desigualdad en renta\\
D)& Con el hundimiento del Titanic, se pone fin a la idea de progreso indefinido.\\
E)& El ideal progresista se origina en el cristianismo y su lucha contra la adversidad.\\ 
\end{tabular}\end{flushleft}%correcta
\begin{key} A
\end{key} 
\begin{hint}
\end{hint}
\begin{answer} Correcta A \\
\end{answer}
\begin{info} %Información técnica de la pregunta
\begin{flushleft}
Eje Temático: HISTORIA: MUNDO, AMÉRICA Y CHILE.\\
Contenido: El impacto de la idea de progreso indefinido en la organización de las sociedades en América y Europa en el siglo XIX. \\
Habilidad: Análisis de fuentes de información,\\
Dificultad: Difícil, muy difícil.\\
Capsula: Por determinar \\
\end{flushleft} 
\end{info}
\end{minipage}\vfill$\;$ %16

\begin{minipage}{\anchopregunta} 
\item Observa la siguiente pintura:\\
\begin{center}
\includegraphics[width=7cm]{imagenes/imagen7.png}
\end{center}
\begin{flushright}
\textit{El progreso Estadounidense}, John Gast, 1871.
\end{flushright}
La idea de progreso se extendió a otros países debido al avance del liberalismo y otros procesos históricos asociados al liberalismo. Basándose en lo anterior, ¿qué elementos de la imagen permiten conectar la expansión de EEUU hacia el oeste con el progreso indefinido? 
\begin{flushleft}\begin{tabular}{@{\hspace{-.001\textwidth}}l@{\hspace{2pt}}p{.965\textwidth}}
A)& La presencia de Germania, quien se encuentra avanzando sobre los indígenas.\\
B)& El mostrar a los indígenas como víctimas del brutal colonialismo americano.\\
C)& El uso de burgueses que avanzan con el fin de poner sus emprendimientos.\\
D)& El uso de inventos industrializados para señalar el progreso y el avance.\\
E)& La domesticación de animales varios, símbolo del progreso científico.\\ 
\end{tabular}\end{flushleft}%correcta
\begin{key} D
\end{key} 
\begin{hint}
\end{hint}
\begin{answer} Correcta D \\
\end{answer}
\begin{info} %Información técnica de la pregunta
\begin{flushleft}
Eje Temático: HISTORIA: MUNDO, AMÉRICA Y CHILE.\\
Contenido: El impacto de la idea de progreso indefinido en la organización de las sociedades en América y Europa en el siglo XIX.
Habilidad: Análisis de fuentes de información,\\
Dificultad: Difícil, muy difícil.\\
\end{flushleft} 
\end{info}
\end{minipage}\vfill$\;$ %17

\begin{minipage}{\anchopregunta}
\item ``El historiador del arte Hans Neuberger analizó 6.500 cuadros pintados entre 1400 y 1967, en 42 museos. Su estudio revela que los pintores del siglo XVIII y principios del XIX solían incluir nubes, que ocupaban entre un 50\% y un 75\% del lienzo. Dos famosos artistas ingleses, John Constable y J. M. W. Turner, cubrían de nubes hasta el 75\% de la imagen. Los cielos habían dejado de ser tan azules como antes, y Neuberger lo atribuyó al tremendo aumento de la contaminación por el carbón. En esa época, los londinenses se asfixiaban con las nubes que sobrevolaban las calles y las viviendas abarrotadas. Algunos artistas pintaban barcos de vapor que remontaban el Támesis bajo una luz amarillenta o grisácea, y la contaminación era un elemento tan esencial en los cuadros como los árboles y los edificios.''\\
\begin{flushright}
\textit{Cuando la niebla de Londres mató a 12.000 personas}, Brian Fegan, El país (2017).
\end{flushright}
Considerando el impacto histórico que ha tenido la Revolución Industrial y lo estudiado por Neuberger, es correcto afirmar que: 
\begin{flushleft}\begin{tabular}{@{\hspace{-.001\textwidth}}l@{\hspace{2pt}}p{.965\textwidth}}
A)& La Revolución Industrial tuvo un temprano inicio en el siglo XV\\
B)& La contaminación era observada desde los inicios de la era industrial. \\
C)& La industrialización era la única causante de la contaminación londinense.\\
D)& Los contaminantes industriales se volvieron habituales y poco dañinos.\\
E)& La contaminación industrial hizo de Londres una ciudad inhabitable.\\ 
\end{tabular}\end{flushleft}%correcta
\begin{key} B
\end{key} 
\begin{hint}
\end{hint}
\begin{answer} Correcta B\\
\end{answer}
\begin{info} %Información técnica de la pregunta
\begin{flushleft}
Eje Temático: Historia: Mundo, América y Chile.\\
Contenido: Características del proceso de industrialización y sus implicancias en diferentes ámbitos de la sociedad en Chile, América y Europa en el siglo XIX.\\
Habilidad: Análisis de fuentes de investigación.\\
Dificultad: Fácil, media.\\
Capsula: Por determinar \\
\end{flushleft} 
\end{info}
\end{minipage}\vfill$\;$ %18

\begin{minipage}{\anchopregunta}
\item Observa la siguiente tabla:
\begin{center}
   \begin{tabular}{|ccc|}
\hline
\multicolumn{3}{|c|}{Ferrocarriles particulares en Chile (1900)}                                                \\ \hline
\multicolumn{1}{|c|}{N°} & \multicolumn{1}{c|}{Nombre de Ferrocarril}                               & Extensión \\ \hline
\multicolumn{1}{|c|}{1}  & \multicolumn{1}{c|}{Ferrocarriles de la Nitrate Railways Comany Limited} & 514 km    \\ \hline
\multicolumn{1}{|c|}{2}  & \multicolumn{1}{c|}{Ferrocarril de Antofagasta.- Sección chilena}        & 442 km    \\ \hline
\multicolumn{1}{|c|}{3}  & \multicolumn{1}{c|}{Ferrocarril de Taltal a Cachinal y ramales}          & 242 km    \\ \hline
\multicolumn{1}{|c|}{4}  & \multicolumn{1}{c|}{Ferrocarril de Copiapo y ramales}                    & 233 km    \\ \hline
\multicolumn{1}{|c|}{5}  & \multicolumn{1}{c|}{Ferrocarril de Carrizal y ramales}                   & 181 km    \\ \hline
\end{tabular}
\end{center}
\begin{flushright}
\textit{Estudios de los ferrocarriles chilenos}, Marín Vicuña. Imprenta Cervantes, Santiago (1900).
\end{flushright}
Considerando el contexto que rodea la implementación de los ferrocarriles en Chile y su rol en la industrialización nacional y mundial, es correcto posible concluir que:
\begin{flushleft}\begin{tabular}{@{\hspace{-.001\textwidth}}l@{\hspace{2pt}}p{.965\textwidth}}
A)& El ferrocarril apoyó el proceso extractivista de Chile en un primer momento, para luego expandirse hacia el sur.\\
B)& Debido al contexto liberal europeo y nacional, los ferrocarriles fueron implementados exclusivamente por privados.\\
C)& Las empresas nacionales monopolizaban los ferrocarriles debido al poco interés de las potencias extranjeras en Chile.\\
D)& Los ferrocarriles, gracias a su alta contaminación, eran vistos como un mal necesario para seguir con el avance económico.\\
E)& La gran extensión de líneas férreas, como en Inglaterra y Estados Unidos, demuestra una avanzada industrialización en Chile.\\ 
\end{tabular}\end{flushleft}%correcta
\begin{key} A
\end{key} 
\begin{hint}
\end{hint}
\begin{answer} Correcta A \\
\end{answer}
\begin{info} %Información técnica de la pregunta
\begin{flushleft}
Eje Temático: HISTORIA: MUNDO, AMÉRICA Y CHILE.\\
Contenido: Características del proceso de industrialización y sus implicancias en diferentes ámbitos de la sociedad en Chile, América y Europa en el siglo XIX.\\
Habilidad: Análisis de fuentes.\\
Dificultad: Media, difícil\\
Capsula: Por determinar \\
\end{flushleft} 
\end{info}
\end{minipage}\vfill$\;$ %19

\begin{minipage}{\anchopregunta}
\item ``[...] En el Estado Libre del Congo, manos congoleñas fueron sistemáticamente amputadas cuando los esclavos africanos no lograron alcanzar las cuotas extractivas de caucho. Los belgas coleccionaban las manos y las disecaban con el fin de preservarlas y contarlas. El proceso fue tan masivo que los habitantes del Congo pensaron que los belgas devoraban manos.\\
Pero los belgas no reconocen su rol en la caída del Congo y, muy a menudo, olvidan el genocidio de aproximadamente 10,000,000 congoleños bajo su mandato colonial. En vez de eso, Bélgica recuerda el colonialismo como un glorioso pasado dentro de dos docenas de grupos conmemorativos.''\\
\begin{flushright}
\textit{El apetito de Amberes por manos africanas,} Jenny Folsom, revista Context vol 15. (2018).
\end{flushright}
El Estado Libre del Congo (1885-1908) fue una colonia personal del rey Leopoldo II de Bélgica. Durante su administración se cometieron atrocidades contra los habitantes de la región como las que se muestran en el texto, un elemento que no fue extraño en otras colonias europeas. Además de eso, el texto presente la siguiente similitud con otros procesos imperialistas de la época:
\begin{flushleft}\begin{tabular}{@{\hspace{-.001\textwidth}}l@{\hspace{2pt}}p{.965\textwidth}}
A)& Los procesos imperialistas, al tener inspiración liberal, eran producto de la iniciativa privada de ciertos individuos.\\
B)& Años luego del proceso, todos los descendientes de los colonizadores consideran el imperialismo como algo positivo.\\
C)& El imperialismo fue un proceso extremadamente brutal, pero sus consecuencias se limitaban a eventos de corto plazo.\\
D)& El Congo, como el resto de países colonizados, fue sometido solamente por potencias europeas en búsqueda de materias primas.\\
E)& Una de las motivaciones tras el castigo físico, y del imperialismo como tal, era la extracción de materias primas.\\ 
\end{tabular}\end{flushleft}%correcta
\begin{key} E
\end{key} 
\begin{hint}
\end{hint}
\begin{answer} Correcta E\\
\end{answer}
\begin{info} %Información técnica de la pregunta
\begin{flushleft}
Eje Temático: HISTORIA: MUNDO, AMÉRICA Y CHILE.\\
Contenido: Las transformaciones culturales, económicas, políticas y territoriales
asociadas al imperialismo europeo del siglo XIX.\\
Habilidad: Análisis de fuentes de información.\\
Dificultad: Fácil, media\\
Capsula: Por determinar\\
\end{flushleft} 
\end{info}
\end{minipage}\vfill$\;$ %20

\begin{minipage}{\anchopregunta}
\item ``En la Exposición Colonial de París, en 1931, se recurrió a 1.500 figurantes para dar vida a diversos pueblos de las colonias francesas, desde el Caribe y el África negra hasta el Asia oriental. De esta forma se esperaba despertar el interés de los más de ocho millones de visitantes por el «primitivismo del mundo no europeo».
Sin embargo, en esta ocasión la respuesta no fue unánimemente favorable. En la prensa, voces anticolonialistas calificaron aquella exposición como «la más espectacular extravagancia colonial montada en Occidente». El escándalo en torno a unos caledonios obligados a interpretar el papel de caníbales obligó a clausurar la atracción y precipitó el fin de este tipo de espectáculos. La ideología que transmitían, sin embargo, encontró otros canales de difusión, en particular el cine, que durante largo tiempo se hizo eco de una propaganda colonialista en la que los pueblos no europeos eran presentados de un modo muy parecido a los zoos humanos del siglo XIX.''\\
\begin{flushright}
\textit{Zoos humanos, la vergüenza de occidente}, National Geographic.
\end{flushright}
Considerando lo postulado por la fuente y sus conocimientos previos, es correcto afirmar que el Imperialismo colonialista: 
\begin{flushleft}\begin{tabular}{@{\hspace{-.001\textwidth}}l@{\hspace{2pt}}p{.965\textwidth}}
A)& Las ideas raciales colonialistas encuentran su fin durante el proceso de descolonización.\\
B)& Los zoológicos humanos obligaban a los colonizados a mostrar todos los aspectos de su cultura.\\
C)& Las potencias construían su imagen de superioridad al mostrar a otros pueblos como inferiores.\\
D)& Se pueden considerar como imperialistas solo a los países donde existieron zoológicos humanos.\\
E)& Los zoológicos humanos iban en contra de los ideales de progreso científico de finales del siglo XIX.\\ 
\end{tabular}\end{flushleft}%correcta
\begin{key} C
\end{key} 
\begin{hint}
\end{hint}
\begin{answer} Correcta C \\
\end{answer}
\begin{info} %Información técnica de la pregunta
\begin{flushleft}
Eje Temático: HISTORIA: MUNDO, AMÉRICA Y CHILE.\\
Contenido: Las transformaciones culturales, económicas, políticas y territoriales
asociadas al imperialismo europeo del siglo XIX.\\
Habilidad: Pensamiento crítico.\\
Dificultad: Media, difícil.\\
Capsula: Por determinar \\
\end{flushleft} 
\end{info}
\end{minipage}\vfill$\;$ %21

\begin{minipage}{\anchopregunta}
\item El periodo histórico nacional que ocurre entre 1823 y 1830 ha tenido diversos nombres: Ensayos Constitucionales, Organización Nacional, Anarquía, entre otros. Por lo tanto, se puede afirmar que no hay consenso historiográfico sobre como categorizar esta etapa histórica. Dicho debate se debe a:\\
\begin{flushleft}\begin{tabular}{@{\hspace{-.001\textwidth}}l@{\hspace{2pt}}p{.965\textwidth}}
A)& Los conflictos que se dieron internacionalmente a pesar del avance legislativo nacional.\\
B)& Un rápido avance económico, pese a los numerosos conflictos políticos de la élite.\\
C)& La existencia de violencia política debido al bandolerismo rural a favor de la monarquía.\\
D)& El acalorado, pero pacífico, debate constitucional dentro de la aristocracia criolla.\\
E)& Las convulsas transformaciones y dificultades varias experimentadas durante la época.\\ 
\end{tabular}\end{flushleft}%correcta
\begin{key} E
\end{key} 
\begin{hint}
\end{hint}
\begin{answer} Correcta E \\
\end{answer}
\begin{info} %Información técnica de la pregunta
\begin{flushleft}
Eje Temático: Historia: Mundo, América y Chile.\\
Contenido: Características políticas y culturales de la formación y consolidación de la República en Chile en el siglo XIX.\\
Habilidad: Pensamiento crítico.\\
Dificultad: Fácil, media. \\
Capsula: Por determinar \\
\end{flushleft} 
\end{info}
\end{minipage}\vfill$\;$ %22

\begin{minipage}{\anchopregunta}
\item ``Portales, a su vez, tampoco podía tolerar el sacudimiento promovido, después de la independencia, por el liberalismo de los pipiolos. Estos países debían estar sometidos al «peso de la noche»; a ese peso de la noche de que habló después Portales, en una carta a don Joaquín Tocornal\\La colonia era el peso de la noche y la emancipación había desencadenado vientos ásperos de contiendas sociales y políticas. Muchos de los opositores patriotas anhelaban separarse de España, pero contaban con seguir en el mismo letargo inconmovible. Portales fué, sin duda, en el fondo revolucionario, separatista como todos, pero a condición de no quebrantar la disciplina monárquica que mantenía la paz, el orden, esa inercia que se desprende del peso de la noche...''\\
\begin{flushright}
\textit{Portales}, Domingo Melfi, Imprenta Universitaria (Santiago, 1930).
\end{flushright}
Diego Portales es uno de los personajes más controvertidos de la historia chilena por su carácter, visión política y rol en la formación del país. Tomando en cuenta sus conocimientos previos y lo descrito por la fuente adjunta, es posible concluir lo siguiente sobre Portales:
\begin{flushleft}\begin{tabular}{@{\hspace{-.001\textwidth}}l@{\hspace{2pt}}p{.965\textwidth}}
A)& Portales era realista, pues consideraba que el peso de la noche colonia era necesario para el buen gobierno.\\
B)& Pese a sus diferencias posteriores con los liberales, Portales deseaba un cambio radical en el gobierno chileno.\\
C)& El conservadurismo portaliano se manifiesta en un apego a las formas coloniales por el orden que trae al país.\\
D)& Portales era un caso especial al anhelar un sistema similar al colonial, pero sin el peso del absolutismo.\\
E)& Para Portales el peso de la noche era algo dependiente del tipo de gobierno, dejando de ser necesario con el tiempo.\\ 
\end{tabular}\end{flushleft}%correcta
\begin{key} C
\end{key} 
\begin{hint}
\end{hint}
\begin{answer} Correcta C \\
\end{answer}
\begin{info} %Información técnica de la pregunta
\begin{flushleft}
Eje Temático: Historia: Mundo, América y Chile.\\
Contenido: Características políticas y culturales de la formación y consolidación de la República en Chile en el siglo XIX.\\
Habilidad: Análisis de fuentes de información.\\
Dificultad: Media.\\
Capsula: Por determinar \\
\end{flushleft} 
\end{info}
\end{minipage}\vfill$\;$ %23

\begin{minipage}{\anchopregunta}
\item ``¿No les ha hervido la sangre cuando el padre inquilino deja a los patrones entrar a nuestro rancho, que no vienen a otra cosa sino a divertirse a costa de la mamá, o las tías, o las hermanas de uno? Claro, él sabe perfectamente que no puedo impedir que ellos ejerzan su derecho de meterse a nuestro rancho y de ``chacotearse con las niñas'', pues, después de todo, junto con nuestra casa, nuestra familia también es como propiedad de ellos''.
\begin{flushright}
Gabriel Salazar, Ser niño huacho en la historia de Chile, 2006.
\end{flushright}
El texto anterior nos describe una escena típica del campo chileno durante el siglo XIX y parte del XX. Ocurre en la Hacienda, la unidad productiva básica del país, debido a su peso en la agricultura. En esta había patrones de gran poder e inquilinos atados a la tierra por contratos que, en resumen, servían como vasallos al patrón. Considerando la anterior, ¿qué información adicional es posible extraer de la fuente?
\begin{flushleft}\begin{tabular}{@{\hspace{-.001\textwidth}}l@{\hspace{2pt}}p{.965\textwidth}}
A)& Debido a la moral católica de los patrones, estos no cometían abusos.\\
B)& Buena parte de los inquilinos se sentían infelices con el trato dado.\\
C)& Solamente los hombres inquilinos sufrían bajo el yugo del patrón.\\
D)& Los inquilinos eran libres de irse cuando quisieran y sin repercusiones.\\
E)& El abuso de los patrones fue corto, limitándose a inicios del siglo XIX.\\
\end{tabular}\end{flushleft}%correcta
\begin{key} E
\end{key} 
\begin{hint}
\end{hint}
\begin{answer} Correcta E\\
\end{answer}
\begin{info} %Información técnica de la pregunta
\begin{flushleft}
Eje Temático: Historia: Mundo, América y Chile.\\
Contenido: Características políticas y culturales de la formación y consolidación de la República en Chile en el siglo XIX.\\
Habilidad: Análisis de fuentes de información.\\
Dificultad: Muy fácil, fácil.\\
Capsula: Por determinar \\
\end{flushleft} 
\end{info}
\end{minipage}\vfill$\;$ %64

\begin{minipage}{\anchopregunta}
\item Observe la siguiente imagen:\\
\begin{center}
    \includegraphics[width=10cm]{imagenes/imagen8.png}
\end{center}
Considerando el contexto independentista y de formación estatal que experimentó el país durante las primeras décadas del siglo XIX, es posible concluir que los cambios en el escudo nacional se produjeron porque:
\begin{flushleft}\begin{tabular}{@{\hspace{-.001\textwidth}}l@{\hspace{2pt}}p{.965\textwidth}}
A)& Los mapuche apoyaron a los españoles tras la Reconquista\\
B)& En Chile se vivía un proceso de búsqueda identitaria.\\
C)& La mayoría de la población chilena era racialmente española.\\
D)& Permitía diferenciarse del indigenismo americanista.\\
E)& Portales reconoció a la nación mapuche como una entidad separada.\\ 
\end{tabular}\end{flushleft}%correcta
\begin{key} B
\end{key} 
\begin{hint}
\end{hint}
\begin{answer} Correcta B \\
\end{answer}
\begin{info} %Información técnica de la pregunta
\begin{flushleft}
Eje Temático: Historia: Mundo, América y Chile.\\
Contenido: Características políticas y culturales de la formación y consolidación de la República en Chile en el siglo XIX.\\
Habilidad: Análisis de fuentes de información.\\
Dificultad: Media.\\
Capsula: Por determinar \\
\end{flushleft} 
\end{info}
\end{minipage}\vfill$\;$ %24

\begin{minipage}{\anchopregunta}
\item ``Señor ministro:\\
Encargado por usted de hacer un viaje al Perú con el objeto de hacer investigaciones relativas a la historia y a la estadística de Chile, que durante algunos siglos formó parte de este virreinato, me he ocupado desde mi llegada en ejecutar tan importante encargo, y tengo el honor de dar a usted una ligera idea de los resultados que he tenido la felicidad de obtener.''
\begin{flushright}
\textit{Carta de Claudio Gay a Mariano Egaña}, Lima (1839).
\end{flushright}
Usando sus conocimientos previos y utilizando el contexto aportado por la fuente, es correcto inferir que durante el periodo conservador:
\begin{flushleft}\begin{tabular}{@{\hspace{-.001\textwidth}}l@{\hspace{2pt}}p{.965\textwidth}}
A)& Se realizó la Guerra contra la Confederación con el fin de obtener fuentes históricas.\\
B)& Los gobiernos conservadores contrataron a expertos internacionales para investigar Chile.\\
C)& Los liberales en el exilio mantenían correspondencia con aquellos que todavía estaban en Chile. \\
D)& Se quería asociar la historia de Chile exclusivamente a su pasado como dominio europeo.\\
E)& Existía un gran interés de datar la historia americana para futuras alianzas internacionales.\\ 
\end{tabular}\end{flushleft}%correcta
\begin{key} B
\end{key} 
\begin{hint}
\end{hint}
\begin{answer} Correcta B \\
\end{answer}
\begin{info} %Información técnica de la pregunta
\begin{flushleft}
Eje Temático: Historia: Mundo, América y Chile.\\
Contenido: Características políticas y culturales de la formación y consolidación de la República en Chile en el siglo XIX.\\
Habilidad: Análisis de fuentes de información.\\
Dificultad: Fácil, media.\\
Capsula: Por determinar \\
\end{flushleft} 
\end{info}
\end{minipage}\vfill$\;$ %25

\begin{minipage}{\anchopregunta}
\item \textbf{1840:}
La empresa naviera Pacific Steam Navegation Company inicia sus servicios en Chile.\\
\textbf{1842:}
Es creada la provincia de Valparaíso, que cuenta con los departamentos de Valparaíso, Quillota y Casablanca.\\
\textbf{1848:}
Se inicia la construcción de los primeros almacenes de la Aduana en Valparaíso.\\
\textbf{1848:}Valparaíso experimenta un crecimiento extraordinario del comercio gracias al auge triguero y la exportación de plata y cobre.\\
\textbf{1850:}Se estable formalmente la Bolsa de Comercio de Valparaíso.\\
\textbf{1852:}Se realizan las obras del ferrocarril de Valparaíso a Santiago.\\
\textbf{1857:}Se funda en Valparaíso el Banco Nacional de Chile, luego de la fusión de los bancos de Valparaíso y Agrícola.\\
\begin{flushright}
\textit{Comercio y finanzas en Valparaíso (1544-1929) (fragmento)} Memoria Chilena.
\end{flushright}
Considerando la información entregada por la cronología, es posible concluir que Valparaíso durante los gobiernos conservadores: 
\begin{flushleft}\begin{tabular}{@{\hspace{-.001\textwidth}}l@{\hspace{2pt}}p{.965\textwidth}}
A)& Solamente era un puerto de intercambio de mercancías por las finanzas centralizadas.\\
B)& Perdió su monopolio sobre el comercio en el Pacífico debido al Canal de Panamá\\
C)& La presencia de los almacenes francos implicó poca actividad por parte de extranjeros.\\
D)& Se convirtió en una fuente importante de ganancias debido a su conectividad mundial.\\
E)& El dinero generado por el puerto permitió una industrialización que rivalizaba con Londres.\\ 
\end{tabular}\end{flushleft}%correcta
\begin{key} D
\end{key} 
\begin{hint}
\end{hint}
\begin{answer} Correcta D \\
\end{answer}
\begin{info} %Información técnica de la pregunta
\begin{flushleft}
Eje Temático: HISTORIA: MUNDO, AMÉRICA Y CHILE\\
Contenido: La inserción de la economía chilena en los procesos de industrialización del mundo atlántico y en los mercados internacionales en el siglo XIX.\\
Habilidad: Pensamiento temporal y espacial.\\
Dificultad: Fácil\\
Capsula: Por determinar \\
\end{flushleft} 
\end{info}
\end{minipage}\vfill$\;$ %26

\begin{minipage}{\anchopregunta}
\item A mediados del siglo XIX se descubrieron importantes yacimientos de oro en Australia y California, dando inicio a la llamada ``Fiebre del oro''. Chile no quedó aislado de dicho proceso pese a la lejanía geográfica, logrando grandes ganancias económicas. A partir de lo anterior, ¿cuál de las siguientes afirmaciones explica el rol de Chile dentro de la Fiebre del oro?:
\begin{flushleft}\begin{tabular}{@{\hspace{-.001\textwidth}}l@{\hspace{2pt}}p{.965\textwidth}}
A)& Proveedor de mano de obra para la extracción aurífera.\\
B)& Abastecedor de insumos en el puerto de Valparaíso.\\
C)& Manufacturador de trenes gracias a Ferrocarriles del Estado.\\
D)& Intermediario entre el Pacífico y el Atlántico por Puerto Montt.\\
E)& Exportador de trigo para suplir la demanda de los mineros.\\ 
\end{tabular}\end{flushleft}%correcta
\begin{key} E
\end{key} 
\begin{hint}
\end{hint}
\begin{answer} Correcta E \\
\end{answer}
\begin{info} %Información técnica de la pregunta
\begin{flushleft}
Eje Temático: Historia: Mundo, América y Chile.\\
Contenido: La inserción de la economía chilena en los procesos de industrialización del mundo atlántico y en los mercados internacionales en el siglo XIX.\\
Habilidad: Pensamiento temporal y espacial.\\
Dificultad: muy fácil\\
Capsula: Por determinar \\\\
\end{flushleft} 
\end{info}
\end{minipage}\vfill$\;$ %27

\begin{minipage}{\anchopregunta}
\item ``[El historiador] Ricardo Krebs dice que «ninguna otra medida causó en aquel tiempo tan profundo impacto y tanto escándalo como la laicización de los cementerios. A los encendidos discursos en el Congreso y a las apasionadas polémicas en la prensa siguieron los actos de violencia: la exhumación de cadáveres en plena noche, la sepultación clandestina, la intervención de la policía. Los católicos se sintieron heridos en sus sentimientos más íntimos y se consideraron perseguidos por “el liberalismo usurpador” (...)».''
\begin{flushright}
\textit{Hacia un nuevo consenso en la regulación de los cementerios: La evolución de las normas civiles y canónicas a lo largo del s. XX.}, Andrés Irarrázaval, Revista Chilena de Derecho (2018).
\end{flushright}
El fragmento anterior toca una de las llamadas ``leyes laicas'' realizadas durante la administración liberal de Domingo Santa María. De acuerdo con la fuente, y considerando sus conocimientos previos, es posible concluir que: 
\begin{flushleft}\begin{tabular}{@{\hspace{-.001\textwidth}}l@{\hspace{2pt}}p{.965\textwidth}}
A)& Las protestas a la ley fueron hechas exclusivamente por el sector político conservador.\\
B)& La ley fue impulsada por los sectores evangélicos que habían llegado recientemente a Chile.\\
C)& La reacción negativa a la ley se produjo porque afectó los valores de la sociedad chilena.\\
D)& Los liberales tenían más apoyo que sus detractores, logrando instaurar un Estado laico.\\
E)& El gobierno de Santa María, debido a las protestas, se vio obligado a derogar la ley de cementerios.\\ 
\end{tabular}\end{flushleft}%correcta
\begin{key} C
\end{key} 
\begin{hint}
\end{hint}
\begin{answer} Correcta C \\
\end{answer}
\begin{info} %Información técnica de la pregunta
\begin{flushleft}
Eje Temático: Historia: Mundo, América y Chile.\\
Contenido: Transformaciones de la sociedad chilena en el tránsito del siglo XIX al siglo XX.\\
Habilidad: Análisis de fuentes de investigación\\
Dificultad: Fácil, media\\
Capsula: Por determinar \\
\end{flushleft} 
\end{info}
\end{minipage}\vfill$\;$ %28

\begin{minipage}{\anchopregunta}
\item ``Lo que fue peculiar al liberalismo chileno, sobre todo en comparación con otros casos hispanoamericanos, es la ausencia de radicalismo y su énfasis en la reforma. Esto no quiere decir que la secularización de la sociedad no fuese drástica, o que no generara fuertes tensiones políticas, sino que las transformaciones fueron realizadas vía reforma, no revoluciones (...).''
\begin{flushright}
\textit{El gobierno y las libertades. La ruta del liberalismo chileno en el siglo XIX}, Ivan Jacksic y Sol Serrano, Revista de Estudios Públicos (2018).
\end{flushright}
De acuerdo con lo planteado por Jacksic y Serrano, y utilizando sus conocimientos previos, es posible afirmar que durante el periodo liberal: 
\begin{flushleft}\begin{tabular}{@{\hspace{-.001\textwidth}}l@{\hspace{2pt}}p{.965\textwidth}}
A)& La revolución fue el método de excelencia parar lograr las reformas más agresivas.\\
B)& Hasta la Guerra Civil de 1891, no hubo conflictos mayores a nivel interno.\\
C)& Debido a su estatus oligárquico, las reformas realizadas solo afectaron a la clase alta.\\
D)& A diferencia del resto de América, las zonas rurales fueron centro de profundas reformas.\\
E)& Estuvo marcado por su paz externa e interna, disminuyendo el rol del ejército.\\ 
\end{tabular}\end{flushleft}%correcta
\begin{key} B
\end{key} 
\begin{hint}
\end{hint}
\begin{answer} Correcta B \\
\end{answer}
\begin{info} %Información técnica de la pregunta
\begin{flushleft}
Eje Temático: Historia: Mundo, América y Chile.\\
Contenido: Características del orden político liberal durante la segunda mitad del siglo XIX en Chile.\\
Habilidad: Análisis de fuentes de información.\\
Dificultad: Media.\\
Capsula: Por determinar \\
\end{flushleft} 
\end{info}
\end{minipage}\vfill$\;$ %29

\begin{minipage}{\anchopregunta}
\item Observe el siguiente gráfico:
\begin{center}
    \includegraphics[width=8cm]{imagenes/imagen9.png}
\end{center}
\begin{flushright}
\textit{Un siglo de historia económica
de Chile 1830-1930.} C. Cariola y O. Sunkel, Ediciones Cultura Hispanica (1982).
\end{flushright}
El grafíco anterior muestra las inversiones de diferentes nacionalidades en la industria salitrera en 1925. Sobre la base de lo expuesto por la fuente y sus conocimientos previos, es posible concluir que:
\begin{flushleft}\begin{tabular}{@{\hspace{-.001\textwidth}}l@{\hspace{2pt}}p{.965\textwidth}}
A)& La inversión inglesa se debió a su imperialismo y la presencia de tropas británicas en Chile.\\
B)& Las inversiones de diferentes países representan, en su mayoría, a capital privado.\\
C)& Las inversiones chilenas en el salitre se deben al control estatal sobre el mismo.\\
D)& Luego de la Primera Guerra Mundial existió un auge salitrero que fomentó la inversión.\\
E)& La inversión chilena en el salitre comienza posteriormente a la Guerra del Pacífico.\\ 
\end{tabular}\end{flushleft}%correcta
\begin{key} B
\end{key} 
\begin{hint}
\end{hint}
\begin{answer} Correcta B\\
\end{answer}
\begin{info} %Información técnica de la pregunta
\begin{flushleft}
Eje Temático: Historia: Mundo, América y Chile.\\
Contenido: Características de las principales transformaciones motivadas por el auge salitrero en Chile fines del siglo XIX y comienzos del siglo XX. \\
Habilidad: Pensamiento temporal y espacial.\\
Dificultad: Media.\\
Capsula: Por determinar \\
\end{flushleft} 
\end{info}
\end{minipage}\vfill$\;$ %30

\begin{minipage}{\anchopregunta}
\item ``El aumento constante de población en el norte de la república y principalmente en la provincia de Antofagasta, cuyas industrias y comercio prosperan con rapidez, tanto por la producción minera y salitrera como por el aumento del comercio con Bolivia, hace que sea necesaria la implantación de fábricas que abaraten los consumos y entreguen con facilidad sus productos al comercio (...).''
\begin{flushright}
\textit{El Industrial (diario antofagastino)} (15/10/1907).
\end{flushright}
Considerando la tesis central planteada por el diario, y usando sus conocimientos previos, es posible plantear el siguiente contrargumento:
\begin{flushleft}\begin{tabular}{@{\hspace{-.001\textwidth}}l@{\hspace{2pt}}p{.965\textwidth}}
A)& Las relaciones con Bolivia seguían mal para 1907, por lo que el comercio era limitado.\\
B)& Al no existir industrias y ser un país exportador, se  dificultaba la industrialización.\\
C)& Las fluctuaciones del valor salitrero  de la época hacían de una inversión industrial algo riesgoso\\
D)& Los trabajadores del salitre tenían un limitado poder adquisitivo debido al pago en fichas.\\
E)& La falta de conectividad con la provincia de Antofagasta impedía un comercio activo.\\ 
\end{tabular}\end{flushleft}%correcta
\begin{key} D
\end{key} 
\begin{hint}
\end{hint}
\begin{answer} Correcta D \\
\end{answer}
\begin{info} %Información técnica de la pregunta
\begin{flushleft}
Eje Temático: Historia: Mundo, América y Chile.\\
Contenido: Características del orden político liberal durante la segunda mitad del siglo XIX en Chile.\\
Habilidad: Pensamiento crítico.\\
Dificultad: Muy difícil.\\
Capsula: Por determinar \\
\end{flushleft} 
\end{info}
\end{minipage}\vfill$\;$ %31

\begin{minipage}{\anchopregunta}
\item Observa la siguiente imagen:
\begin{center}
    \includegraphics[width=7cm]{imagenes/imagen17.png}\\
Mapa de Chile en 1810 y mapa de Chile en 1883.
\end{center}
El mapa anterior, de todas formas, tiene ligeros errores. Principalmente en que tanto abarca Chile, pues para 1810 no existían ciudades en el Estrecho de Magallanes o el hecho de que seguían existiendo españoles en la Isla de Chiloé. Esto se debe a que decir que un territorio te pertenece es diferente a tener una muestra de soberanía sobre este (como un fuerte, por ejemplo). Considerando lo anterior, ¿de qué manera se puede relacionar la pérdida de soberanía con la pérdida de la Patagonia?:
\begin{flushleft}\begin{tabular}{@{\hspace{-.001\textwidth}}l@{\hspace{2pt}}p{.965\textwidth}}
A)&  El texto con más antigüedad que afirmaba la chilenidad de la Patagonia era la Constitución de 1833, siendo demasiado reciente.\\
B)& Los argentinos invadieron violentamente la Patagonia, y como Chile estaba en guerra, no fue capaz de defender su territorio.\\
C)& La colonización del Llanquihue siguió a través de la cordillera, pero los alemanes no eran leales al Estado, dejando la Patagonia.\\
D)& La Patagonia chilena había sido ignorada durante décadas, por lo tanto, no existía población chilena que ejerciera soberanía.\\
\end{tabular}\end{flushleft}%correcta
\begin{key} D
\end{key} 
\begin{hint}
\end{hint}
\begin{answer} Correcta D\\
\end{answer}
\begin{info} %Información técnica de la pregunta
\begin{flushleft}
Eje Temático: Historia: Mundo, América y Chile.\\
Contenido: Transformaciones de la sociedad chilena en el tránsito del siglo XIX al siglo XX. \\
Habilidad: Pensamiento temporal y espacial.\\
Dificultad: Media\\
Capsula: Por determinar \\
\end{flushleft}
\end{info}
\end{minipage}\vfill$\;$ %65

\begin{minipage}{\anchopregunta}
\item ``Detrás de su aparente estabilidad, a la vuelta de siglo, Chile se caracterizó por un extremo contraste entre el desarrollo nacional que, por una parte, trajo prosperidad económica (...) y, por otra, nuevas formas de pobreza urbana (...). Es decir, que la migración y la urbanización vinculada con la particular forma de desarrollo económico de Chile (...) incrementaron los niveles de pobreza urbana (...). Así, este período de extrema estabilidad política se caracterizó también por el clamor público (...) por la cuestión social.''
\begin{flushright}
\textit{Labores propias de su sexo. Género, políticas y trabajo en Chile urbano 1900-1930}, Elizabeth Hutchison, LOM (2006).
\end{flushright}
La sociedad chilena presentó una serie de cambios durante el periodo finisecular, como los retratados por la fuente, pero también presentó continuidades con el Chile del siglo XIX. Considerando lo postulado por la fuente y sus conocimientos previos, es posible escoger la siguiente continuidad como correcta:
\begin{flushleft}\begin{tabular}{@{\hspace{-.001\textwidth}}l@{\hspace{2pt}}p{.965\textwidth}}
A)& Las relaciones inquilinaje presentes en las zonas rurales.\\
B)& Un régimen dominado por la misma burguesía formada en la colonia.\\
C)& Conflictos políticos entre la clase alta que se tornan violentos.\\SS
D)& Un modelo exportador de materias primas enfocado en trigo.\\
E)& La desigualdad económica ocasionada por la Cuestión Social.\\ 
\end{tabular}\end{flushleft}%correcta
\begin{key} A
\end{key} 
\begin{hint}
\end{hint}
\begin{answer} Correcta A \\
\end{answer}
\begin{info} %Información técnica de la pregunta
\begin{flushleft}
Eje Temático: Historia: Mundo, América y Chile.\\
Contenido: Transformaciones de la sociedad chilena en el tránsito del siglo XIX al siglo XX. \\
Habilidad: Pensamiento temporal y espacial.\\
Dificultad: Fácil, media\\
Capsula: Por determinar \\
\end{flushleft} 
\end{info}
\end{minipage}\vfill$\;$ %32

\begin{minipage}{\anchopregunta}
\item ``Única colectividad femenina en Chile que trabaja directamente por la obtención de los derechos civiles, jurídicos y políticos de la mujer. ¡Damas, pasad a inscribiros hoy mismo en los registros de este Partido! Esposas, madres, si sois indiferentes por vuestro porvenir no tenéis derecho a serlo por el de vuestras hijas. Venid a nuestras filas a laborar la felicidad de la mujer chilena.''
\begin{flushright}
\textit{Acción Femenina} Año 1, N°1 (1922).
\end{flushright}
Durante el cambio de siglo las mujeres empezaron a tomar un rol más activo en la sociedad chilena, siendo la fuente anterior una prueba de ello. Además, considerando sus conocimientos previos, es correcto concluir sobre dicho proceso:
\begin{flushleft}\begin{tabular}{@{\hspace{-.001\textwidth}}l@{\hspace{2pt}}p{.965\textwidth}}
A)& El objetivo de esta primera ola feminista era la obtención de la igualdad total con el hombre.\\
B)& En Chile se dio a través de la obtención del derecho a voto durante la República Parlamentaria.\\
C)& Las ideas feministas fueron apoyadas unánimemente por mujeres de todos los sectores políticos.\\
D)& Las feministas tendían a estar en zonas rurales y pertenecían a diversos rangos etarios.\\
E)& Fue un proceso que se dio en Occidente debido a la incorporación de la mujer al mercado laboral.\\ 
\end{tabular}\end{flushleft}%correcta
\begin{key} E
\end{key} 
\begin{hint}
\end{hint}
\begin{answer} Correcta E \\
\end{answer}
\begin{info} %Información técnica de la pregunta
\begin{flushleft}
Eje Temático: Historia: Mundo, América y Chile.\\
Contenido: Transformaciones de la sociedad chilena en el tránsito del siglo XIX al siglo XX. \\
Habilidad: Pensamiento temporal y espacial.\\
Dificultad: Fácil.\\
Capsula: Por determinar \\
\end{flushleft} 
\end{info}
\end{minipage}\vfill$\;$ %33


\begin{minipage}{\anchopregunta}
\item ``Caso cuatro. 39 años de edad. Gaseado el 29 de julio de 1917. Admitido en el hospital de campaña el mismo día. Muerte unos diez días después. Pigmentación pardusca presente en grandes áreas del cuerpo. Un anillo blanco de piel en el lugar donde estaba el reloj de pulsera. Marcadas quemaduras superficiales en cara y escroto. Laringe muy congestionada. Toda la tráquea cubierta de una membrana amarilla. Bronquios contienen abundante gas. Pulmones muy voluminosos. Pulmón derecho muestra gran colapso en la base. Hígado congestionado y graso. Estómago muestra numerosas hemorragias submucosas. Sustancia cerebral excesivamente húmeda y muy congestionada.''
\begin{flushright}
\textit{Informe post-mortem de los historiales médicos oficiales de soldados británicos} En Smith, G. M., Casualties and Medical Statistics of the Great War. Londres: Ed. Museo Imperial de la Guerra, 1997.
\end{flushright}
La Primera Guerra Mundial es conocida como la primera guerra moderna. Por lo mismo, es correcto que la fuente señala el siguiente cambio en la forma de hacer guerras:
\begin{flushleft}\begin{tabular}{@{\hspace{-.001\textwidth}}l@{\hspace{2pt}}p{.965\textwidth}}
A)& La idea de guerra total debido al aumento del número de víctimas civiles.\\
B)& La industrialización de la muerte al crear campos de exterminio especializados.\\
C)& La intensidad del desgaste mental, por la aparición de enfermedades mentales en soldados.\\
D)& La unión de ciencia y guerra al aplicar los avances del siglo XIX en armamento.\\
E)& La violencia bélica, pues la modernidad es la que trae formas tortuosas de morir.\\ 
\end{tabular}\end{flushleft}%correcta
\begin{key} D
\end{key} 
\begin{hint}
\end{hint}
\begin{answer} Correcta D \\
\end{answer}
\begin{info} %Información técnica de la pregunta
\begin{flushleft}
Eje Temático: Historia: Mundo, América y Chile.\\
Contenido: Los impactos geopolíticos y sociales de la Primera Guerra Mundial\\
Habilidad: Análisis de fuente\\
Dificultad: Fácil, media.\\
Capsula: Por determinar \\
\end{flushleft} 
\end{info}
\end{minipage}\vfill$\;$ %34

\begin{minipage}{\anchopregunta}
\item Observe los siguientes mapas:
\begin{center}
    \includegraphics[width=12cm]{imagenes/imagen10.png}
\end{center}
Considerando la información aportada y sus conocimientos previos, es correcto afirmar que la Primera Guerra Mundial trajo como consecuencia:
\begin{flushleft}\begin{tabular}{@{\hspace{-.001\textwidth}}l@{\hspace{2pt}}p{.965\textwidth}}
A)& La desintegración de los imperios pertenecientes a las Potencias Centrales.\\
B)& La formación de países más pequeños que resolverán el problema del nacionalismo.\\
C)& La vuelta al poder de los gobernantes que habían sido expulsados por el conflicto.\\
D)& La creación de países aliados de la Unión Soviética en su frontera occidental.\\
E)& Los países de la Triple Entente terminaron la guerra sin ganancias territoriales.\\ 
\end{tabular}\end{flushleft}%correcta
\begin{key} A
\end{key} 
\begin{hint}
\end{hint}
\begin{answer} Correcta A \\
\end{answer}
\begin{info} %Información técnica de la pregunta
\begin{flushleft}
Eje Temático: Historia: Mundo, América y Chile.\\
Contenido: Los impactos geopolíticos y sociales de la Primera Guerra Mundial\\
Habilidad: Análisis de fuente\\
Dificultad: Fácil, media.\\
Capsula: Por determinar \\
\end{flushleft} 
\end{info}
\end{minipage}\vfill$\;$ %35

\begin{minipage}{\anchopregunta}
\item ``El hombre fascista es el individuo que es nación y patria, ley moral que une a los individuos y a las generaciones en una tradición y en una misión, que suprime el instinto de la vida encerrada en el reducido límite del placer para instaurar en el deber una vida superior, libre de límites de espacio y de tiempo: una vida en la cual el individuo, en virtud de su abnegación, del sacrificio de sus intereses particulares, y aún de su misma muerte, realiza aquella existencia, totalmente espiritual, en la que consiste su valor de hombre.''
\begin{flushright}
\textit{Discurso de Benito Musolinni}, 1932.
\end{flushright}
Los gobiernos totalitarios surgieron durante la primera mitad del siglo XX y tocaron los extremos de ambos sectores políticos. Pese a ello, usando sus conocimientos previos y la información entregada por la fuente, es posible afirmar que los gobiernos totalitarios tienen la siguiente característica en común y que los diferencia de otras formas de gobierno:
\begin{flushleft}\begin{tabular}{@{\hspace{-.001\textwidth}}l@{\hspace{2pt}}p{.965\textwidth}}
A)& Un ideal imperialista con asociación al pasado romano.\\
B)& Un profundo rechazo hacia los sistemas democráticos.\\
C)& Un control total sobre lo público y la vida privada.\\
D)& Un fuerte nacionalismo y rechazo los grupos marginados.\\
E)& Un sentimiento anticomunista y anticapitalista.\\ 
\end{tabular}\end{flushleft}%correcta
\begin{key} C
\end{key} 
\begin{hint}
\end{hint}
\begin{answer} Correcta C \\
\end{answer}
\begin{info} %Información técnica de la pregunta
\begin{flushleft}
Eje Temático: HISTORIA: MUNDO, AMÉRICA Y CHILE.\\
Contenido: Nuevos modelos políticos y económicos derivados de la crisis del Estado liberal a comienzos del siglo XX: totalitarismos europeos, populismo en América Latina e inicios del Estado de bienestar.\\
Habilidad: Pensamiento temporal y espacial.\\
Dificultad: Fácil, media.\\
Capsula: Por determinar \\
\end{flushleft} 
\end{info}
\end{minipage}\vfill$\;$ %36

\begin{minipage}{\anchopregunta}
\item Lee el siguiente poema:
\begin{center}
\begin{tabular}{ll}
\begin{tabular}[c]{@{}l@{}}Los que vivís seguros\\ En vuestras casas caldeadas\\ Los que os encontráis, al volver por la tarde,\\ La comida caliente y los rostros amigos:\\ Considerad si es un hombre\\ Quien trabaja en el fango\\ Quien no conoce la paz\\ Quien lucha por la mitad de un panecillo\\ Quien muere por un sí o por un no.\\ Considerad si es una mujer\\ Quien no tiene cabellos ni nombre\\ Ni fuerzas para recordarlo \end{tabular} & \begin{tabular}[c]{@{}l@{}}Vacía la mirada y frío el regazo\\ Como una rana invernal\\ Pensad que esto ha sucedido:\\ Os encomiendo estas palabras.\\ Grabadlas en vuestros corazones\\ Al estar en casa, al ir por la calle,\\ Al acostaros, al levantaros;\\ Repetídselas a vuestros hijos.\\ O que vuestra casa se derrumbe,\\ La enfermedad os imposibilite,\\ Vuestros descendientes os vuelvan el rostro.\end{tabular}
\end{tabular}
\end{center}
\begin{flushright}
\textit{Si esto es un hombre} Primo Levi, 1956.
\end{flushright}
Primo Levi fue un partisano italiano de origen judío. Deportado en 1944 al campo de concentración de Monowitz su trilogía sobre el Holocausto es uno de los análisis más importantes sobre la \textit{Shoa} y forman parte de las obras más importantes de todo el siglo XX. Considerando el contexto biográfico del autor y el tiempo en el que vivió, ¿qué mensaje desea entregar con el poema?
\begin{flushleft}\begin{tabular}{@{\hspace{-.001\textwidth}}l@{\hspace{2pt}}p{.965\textwidth}}
A)& Una consideración sobre aquellos que estaban sufriendo al momento de escribir el poema.\\
B)& Un recordatorio sobre el genocidio para aquellos que no lo vivieron y para el futuro.\\
C)& Un mensaje de optimismo debido al término de la guerra y los posteriores juicios.\\
D)& Una condena al pueblo alemán por ser cómplices del exterminio del pueblo judío.\\
E)& Una descripción de la vida cotidiana de hombres y mujeres en los campos de concentración.\\ 
\end{tabular}\end{flushleft}%correcta
\begin{key} B
\end{key} 
\begin{hint}
\end{hint}
\begin{answer} Correcta B \\
\end{answer}
\begin{info} %Información técnica de la pregunta
\begin{flushleft}
Eje Temático: HISTORIA: MUNDO, AMÉRICA Y CHILE.\\
Contenido: Nuevos modelos políticos y económicos derivados de la crisis del Estado liberal a comienzos del siglo XX: totalitarismos europeos, populismo en América Latina e inicios del Estado de bienestar.\\
Habilidad: Análisis de fuentes.\\
Dificultad: Media, difícil.\\
Capsula: Por determinar \\
\end{flushleft} 
\end{info}
\end{minipage}\vfill$\;$ %58

\begin{minipage}{\anchopregunta}
\item Observe el siguiente afiche:\\
\begin{center}
    \includegraphics[width=7cm]{imagenes/imagen11.png}
\end{center}
\begin{flushright}
\textit{Publicidad de la película antisemita ``El eterno judío''}, Alemania, 1940.
\end{flushright}
Utilizando sus conocimientos previos para analizar la fuente, es correcto concluir que la propaganda nazi:
\begin{flushleft}\begin{tabular}{@{\hspace{-.001\textwidth}}l@{\hspace{2pt}}p{.965\textwidth}}
A)& Era particular, pues solo ellos utilizaban las películas de forma política.\\
B)& Se enfocaba solamente en el antisemitismo para articular su política.\\
C)& Utilizaba cualquier medio disponible para llegar masivamente a la población.\\
D)& Empleaba un lenguaje complejo que solo entendían los miembros del partido.\\
E)& Se valía de ideas nuevas, como el antisemitismo, para propagarse.\\ 
\end{tabular}\end{flushleft}%correcta
\begin{key} C
\end{key} 
\begin{hint}
\end{hint}
\begin{answer} Correcta C\\
\end{answer}
\begin{info} %Información técnica de la pregunta
\begin{flushleft}
Eje Temático: HISTORIA: MUNDO, AMÉRICA Y CHILE.\\
Contenido: Nuevos modelos políticos y económicos derivados de la crisis del Estado liberal a comienzos del siglo XX: totalitarismos europeos, populismo en América Latina e inicios del Estado de bienestar.\\
Habilidad: Análisis de fuentes.\\
Dificultad: Media.\\
Capsula: Por determinar \\
\end{flushleft} 
\end{info}
\end{minipage}\vfill$\;$ %37

\begin{minipage}{\anchopregunta}
\item La Organización de las Naciones Unidas se creó después de la Segunda Guerra Mundial como garante de la paz. No fue el primer organismo que trataba de lograr dicho objetivo, pues le precede la fallida Sociedad de las Naciones. Por ello, es posible afirmar que un elemento que diferencia a la ONU de la Sociedad de las Naciones es:
\begin{flushleft}\begin{tabular}{@{\hspace{-.001\textwidth}}l@{\hspace{2pt}}p{.965\textwidth}}
A)& El uso de sanciones económicas en contra de los países que atenten contra la paz.\\
B)& La expulsión de los miembros que cometan violaciones a los derechos humanos.\\
C)& La elección de un bando e intervención directa en conflictos internacionales.\\
D)& La adopción y mejora de los Derechos Humanos creados por la Sociedad de las Naciones.\\
E)& La inclusión de las grandes potencias ganadoras del conflicto que le dio origen.\\ 
\end{tabular}\end{flushleft}%correcta
\begin{key} E
\end{key} 
\begin{hint}
\end{hint}
\begin{answer} Correcta E \\
\end{answer}
\begin{info} %Información técnica de la pregunta
\begin{flushleft}
Eje Temático: HISTORIA: MUNDO, AMÉRICA Y CHILE.\\
Contenido: La configuración de un nuevo orden mundial de posguerra, el inicio de procesos de descolonización y la creación de un nuevo marco regulador de las relaciones internacionales (Organización de las Naciones Unidas y Declaración Universal de los Derechos Humanos).\\
Habilidad: Pensamiento temporal y espacial.\\
Dificultad: Media.\\
Capsula: Por determinar \\
\end{flushleft} 
\end{info}
\end{minipage}\vfill$\;$ %38

\begin{minipage}{\anchopregunta}
\item Observa el siguiente mapa:
\begin{center}
\includegraphics[width=15cm]{imagenes/imagen12.jpg}
\end{center}
De acuerdo a la información entregada por el mapa más sus conocimientos previos, es posible señalar que:
\begin{flushleft}\begin{tabular}{@{\hspace{-.001\textwidth}}l@{\hspace{2pt}}p{.965\textwidth}}
A)& El proceso de descolonización trajo una época de paz para todos los países independizados.\\
B)& El agotamiento de recursos en las colonias fue la principal razón de su abandono por la metrópolis.\\
C)& La principal característica de los países del tercer mundo es su independencia de Inglaterra.\\
D)& Estados Unidos y la Unión Soviética apoyaron la descolonización debido a razones ideológicas. \\
E)& La descolonización creo una crisis económica que trajo el fin de la época dorada del capitalismo.\\ 
\end{tabular}\end{flushleft}%correcta
\begin{key} D
\end{key} 
\begin{hint}
\end{hint}
\begin{answer} Correcta D \\
\end{answer}
\begin{info} %Información técnica de la pregunta
\begin{flushleft}
Eje Temático: HISTORIA: MUNDO, AMÉRICA Y CHILE.\\
Contenido: La configuración de un nuevo orden mundial de posguerra, el inicio de procesos de descolonización y la creación de un nuevo marco regulador de las relaciones internacionales (Organización de las Naciones Unidas y Declaración Universal de los Derechos Humanos).\\
Habilidad: Pensamiento temporal y espacial.\\
Dificultad: Media, difícil.\\
Capsula: Por determinar \\
\end{flushleft} 
\end{info}
\end{minipage}\vfill$\;$ %39

\begin{minipage}{\anchopregunta}
\item ``Ranchos y más ranchos. Construidos con latas, desechos de ladrillos, con tablas podridas, con alambres: ranchos construidos con muerte, con muerte venida de no se sabe dónde, con cadáveres de materiales que alguna vez fueron dignos guardadores de la pasión humana. Calles, calles y más calles, sinuosas, caprichosas, igual que la entrecortada ilusión de gentes humildes. Luego, chiquillos, niños de triste alegría, sin volantines, sin ñeclas, pero con tarritos en donde vacían tiernamente arena, ripio y desperdicios, como harían en una playa muerta''\\
\begin{flushright}
\textit{La sangre y la esperanza} Nicomedes Guzmán, 1943.
\end{flushright}
El extracto anterior retrata la perspectiva de Nicomedes Guzmán respecto a las poblaciones callampas. A pesar de que el proceso de migración campo-ciudad no era algo nuevo para la época, la fuente permite diferencias las poblaciones callampas de los conventillos al plantear que:
\begin{flushleft}\begin{tabular}{@{\hspace{-.001\textwidth}}l@{\hspace{2pt}}p{.965\textwidth}}
A)& Los habitantes de los conventillos tenían una vida más lujosa.\\
B)& Las poblaciones callampa surgían solamente en la ribera del Mapocho.\\
C)& La construcción de las poblaciones callampas eran informales.\\
D)& La prostitución era un problema de las poblaciones callampa.\\
E)& Los conventillos se construían en terrenos baldíos y rurales.\\ 
\end{tabular}\end{flushleft}%correcta
\begin{key} C
\end{key} 
\begin{hint}
\end{hint}
\begin{answer} Correcta C \\
\end{answer}
\begin{info} %Información técnica de la pregunta
\begin{flushleft}
Eje Temático: HISTORIA: MUNDO, AMÉRICA Y CHILE.\\
Contenido: Caracterización de la sociedad chilena de mediados del siglo XX, considerando la extendida pobreza y su precariedad, y los impactos de la migración del campo a la ciudad. \\
Habilidad: Análisis de fuentes.\\
Dificultad: Fácil.\\
Capsula: Por determinar \\
\end{flushleft} 
\end{info}
\end{minipage}\vfill$\;$ %40

\begin{minipage}{\anchopregunta}
\item Durante todo el siglo XX en Chile se produjeron una serie de transformaciones sociales de gran impacto a nivel nacional. Pese a ello, hubo un sector de la población que se mantuvo al margen de dichos cambios hasta la década de 1960. Dicho grupo es:
\begin{flushleft}\begin{tabular}{@{\hspace{-.001\textwidth}}l@{\hspace{2pt}}p{.965\textwidth}}
A)& Las mujeres, ya que solo se vieron afectadas por el voto femenino, que no tuvo impacto en las votaciones nacionales.\\
B)& Los inmigrantes, pues la inmigración es un proceso que se solo pasó durante el siglo XIX y en la actualidad.\\
C)& Los grupos terratenientes, ya que la migración campo-ciudad acabó con su poder político sobre el campesino.\\
D)& Las clases medias, pues fueron disminuyendo en número a medida que se profundizó la crisis del modelo ISI.\\
E)& El campesinado, pues el campo no sufrió ningún cambio significativo, incluso con la aplicación del modelo ISI.\\ 
\end{tabular}\end{flushleft}%correcta
\begin{key} E
\end{key} 
\begin{hint}
\end{hint}
\begin{answer} Correcta E \\
\end{answer}
\begin{info} %Información técnica de la pregunta
\begin{flushleft}
Eje Temático: \\
Contenido: Caracterización de la sociedad chilena de mediados del siglo XX, considerando la extendida pobreza y su precariedad, y los impactos de la migración del campo a la ciudad.\\
Habilidad: Pensamiento temporal y espacial.\\
Dificultad: Muy fácil.\\
Capsula: Por determinar \\
\end{flushleft} 
\end{info}
\end{minipage}\vfill$\;$ %41

\begin{minipage}{\anchopregunta}
\item ``En sentido estricto la Unidad Popular buscaba realizar transformaciones profundas en la esfera de la producción, modificando la propiedad, ‘sin tomar el poder’, sin una revolución política, negando pero también superando en la práctica a la teoría bolchevique. [...] Pero la mayoría de los sujetos sociales (partidos, organizaciones, comités, personalidades o ciudadanos) vivieron la experiencia de la Unidad Popular como si fuera una revolución socialista que, aunque se ejecutaba desde el Estado o desde arriba, iba a tener en la lucha política todos los efectos polarizadores de una revolución socialista a secas.''
\begin{flushright}
\textit{La vía chilena al socialismo} Tomás Moulian, 2005.
\end{flushright}
En base a lo planteado por la fuente, ¿Cuál de los siguientes factores es causa de la inestabilidad política durante el gobierno de Allende?
\begin{flushleft}\begin{tabular}{@{\hspace{-.001\textwidth}}l@{\hspace{2pt}}p{.965\textwidth}}
A)& La falta de apoyo de la Democracia Cristiana para poder lograr un proyecto claro y moderado.\\
B)& El aumento de las tensiones a nivel país por un proyecto que era interpretado de diversas maneras.\\
C)& La presencia de políticas subversivas en los grupos de derecha financiados por la CIA y empresarios.\\
D)& La presencia de políticas marxistas que pasaban por encima del orden democrático establecido.\\
E)& El apoyo recibido de países comunistas extranjeras al gobierno con el fin de realizar un autogolpe. \\ 
\end{tabular}\end{flushleft}%correcta
\begin{key} B
\end{key} 
\begin{hint}
\end{hint}
\begin{answer} Correcta B\\
\end{answer}
\begin{info} %Información técnica de la pregunta
\begin{flushleft}
Eje Temático: HISTORIA: MUNDO, AMÉRICA Y CHILE.\\
Contenido: Distintas interpretaciones historiográficas del golpe de Estado de 1973 y  el quiebre de la democracia en Chile.\\
Habilidad: Análisis de fuentes de información\\
Dificultad: Fácil\\
Capsula: Por determinar \\
\end{flushleft} 
\end{info}
\end{minipage}\vfill$\;$ %42

\begin{minipage}{\anchopregunta}
\item ``Algunas de las dificultades del PDC, sin duda, provenían de la abrumadora dimensión de su triunfo en 1964-1965. Su decisión de gobernar por sí solo, sin aliados, era aberrante para las normas habituales de la política chilena. Las actitudes algo triunfalistas, por no decir arrogantes, [...] difícilmente iban a traerles amigos. [Los] tratos y concesiones de la política de coalición habían sido olvidados por este Partido recién llegado y con un éxito repentino entre los partidos tradicionales''\\
\begin{flushright}
\textit{Historia de Chile, 1808-1994}, Simon Collier y  William Sater, Madrid: Cambridge University Press (1998).
\end{flushright}
La fuente anterior presenta parte del acontecer político durante el gobierno de Eduardo Frei. Considerando el contexto político chileno durante los gobiernos de Frei y Allende y utilizando la información entregada por la fuente, es correcto concluir que el siguiente antecedente de la inestabilidad política durante la UP se formó durante el gobierno del demócrata cristiano:
\begin{flushleft}\begin{tabular}{@{\hspace{-.001\textwidth}}l@{\hspace{2pt}}p{.965\textwidth}}
A)& La arrogancia de los partidos de izquierda al conseguir grandes resultados electorales en los 60's.\\
B)& La ausencia del centro político como figura moderada, dialogante y articuladora de coaliciones.\\
C)& La perdida de poder por parte de los radicales en pos de una Democracia Cristiana violenta.\\
D)& La crisis inflacionaria producto de las malas decisiones tomadas durante el gobierno de Frei.\\
E)& La radicalización de los extremos ideológicos debido al contexto internacional de la Guerra Fría.\\ 
\end{tabular}\end{flushleft}%correcta
\begin{key} B
\end{key} 
\begin{hint}
\end{hint}
\begin{answer} Correcta B \\
\end{answer}
\begin{info} %Información técnica de la pregunta
\begin{flushleft}
Eje Temático: HISTORIA: MUNDO, AMÉRICA Y CHILE.\\
Contenido: Distintas interpretaciones historiográficas del golpe de Estado de 1973 y el quiebre de la democracia en Chile.\\
Habilidad: Análisis de fuentes.\\
Dificultad: Media, difícil.\\
Capsula: Por determinar \\
\end{flushleft} 
\end{info}
\end{minipage}\vfill$\;$ %43

\begin{minipage}{\anchopregunta}
\item Observa el siguiente gráfico:
\begin{center}
\includegraphics[width=13cm]{imagenes/imagen13.png}
\end{center}
Considerando la información entregada por la figura anterior y su conocimiento previo, es correcto concluir que:
\begin{flushleft}\begin{tabular}{@{\hspace{-.001\textwidth}}l@{\hspace{2pt}}p{.965\textwidth}}
A)& Para el funcionamiento del sistema neoliberal era necesario eliminar la preponderancia del sector público.\\
B)& La dictadura de Pinochet privatizó todas las empresas del sector público, incluyendo servicios básicos.\\
C)& El periodo de crisis experimentado desde 1982 se debió a la poca recaudación por la privatización de empresas.\\
D)& Pese al avance del neoliberalismo en los gobiernos posteriores, la privatización de empresas se detuvo.\\
E)& Las empresas privatizadas tuvieron una baja rentabilidad, por lo que actualmente no se encuentran operando.\\ 
\end{tabular}\end{flushleft}%correcta
\begin{key} A
\end{key} 
\begin{hint}
\end{hint}
\begin{answer} Correcta A\\
\end{answer}
\begin{info} %Información técnica de la pregunta
\begin{flushleft}
Eje Temático: HISTORIA: MUNDO, AMÉRICA Y CHILE.\\
Contenido: Características del modelo económico neoliberal implementado en Chile durante la Dictadura Militar.\\
Habilidad: Análisis de fuente.\\
Dificultad: Fácil, media.\\
Capsula: Por determinar \\
\end{flushleft} 
\end{info}
\end{minipage}\vfill$\;$ %44

\begin{minipage}{\anchopregunta}
\item Observa el siguiente gráfico:
\begin{center}
\includegraphics[width=15cm]{imagenes/imagen14.png}
\end{center}
Usando su conocimientos previos, ¿cuál es una posible explicación para el aumento del comercio internacional?
\begin{flushleft}\begin{tabular}{@{\hspace{-.001\textwidth}}l@{\hspace{2pt}}p{.965\textwidth}}
A)& Las buenas relaciones internacionales existentes con el gobierno de Pinochet.\\
B)& La existencia de planes de ayuda entre dictaduras, como la Operación Cóndor.\\
C)& El aumento continuado del precio del cobre tras la crisis económica de 1973.\\
D)& La diversificación de las exportaciones y el uso de ventajas comparativas.\\
E)& La rápida industrialización del sector privado para exportar tecnología de punta.\\ 
\end{tabular}\end{flushleft}%correcta
\begin{key} D
\end{key} 
\begin{hint}
\end{hint}
\begin{answer} Correcta D\
\end{answer}
\begin{info} %Información técnica de la pregunta
\begin{flushleft}
Eje Temático: HISTORIA: MUNDO, AMÉRICA Y CHILE.\\
Contenido: Características del modelo económico neoliberal implementado en Chile durante la Dictadura Militar.\\
Habilidad: Pensamiento temporal y espacial.\\
Dificultad: Fácil.\\
Capsula: Por determinar \\
\end{flushleft} 
\end{info}
\end{minipage}\vfill$\;$ %45

\begin{minipage}{\anchopregunta}
\item ``Artículo 1°- Concédese amnistía a todas las personas que, en calidad de autores, cómplices o encubridores hayan incurrido en hechos delictuosos, durante la vigencia de la situación de Estado de Sitio, comprendida entre el 11 de septiembre de 1973 y el 10 de marzo de 1978, siempre que no se encuentren actualmente sometidas a proceso o condenadas.
Artículo 2°- Amnistíase, asimismo, a las personas que a la fecha de vigencia del presente decreto ley se encuentren condenadas por tribunales militares, con posterioridad al 11 de septiembre de 1973.''\\
\begin{flushright}
Decreto Ley n° 2.191, promulgado el 18 de abril de 1978.
\end{flushright}
Considerando las violaciones a los Derechos Humanos cometidas por el régimen de Pinochet, ¿qué función cumplía la ley de amnistía?
\begin{flushleft}\begin{tabular}{@{\hspace{-.001\textwidth}}l@{\hspace{2pt}}p{.965\textwidth}}
A)& Procuraba bajar las tensiones con los grupos de izquierda.\\
B)& Protegía a los agentes del Estado que violaban los DD.HH.\\
C)& Buscaba una transición armónica hacia un gobierno democrático.\\
D)& Daba a conocer la política de violación de DD.HH. al público.\\
E)& Impedía el actuar de la Viacaría de la Solidardad.\\ 
\end{tabular}\end{flushleft}%correcta
\begin{key} B
\end{key} 
\begin{hint}
\end{hint}
\begin{answer} Correcta B \\
\end{answer}
\begin{info} %Información técnica de la pregunta
\begin{flushleft}
Eje Temático: HISTORIA: MUNDO, AMÉRICA Y CHILE.\\
Contenido: Las violaciones a los derechos humanos y la supresión del Estado de derecho durante la Dictadura Militar.\\
Habilidad: Pensamiento crítico.\\
Dificultad: Media.\\
Capsula: Por determinar \\
\end{flushleft} 
\end{info}
\end{minipage}\vfill$\;$ %46

\begin{minipage}{\anchopregunta}
\item \begin{flushleft}
    ``Respetado señor ministro:
\end{flushleft}
El pasado 17 de mayo de 1985, el suscrito, puso en su conocimiento diversas amenazas y actos cometidos contra funcionarios que, en nuestra opinión, significaban un peligro para la vida e integridad física de ellos y sus familias. [...] Estos sucesos no ocurren de forma asilada: ello se han repetido, en forma ininterrumpida, desde que se iniciara el año. No ocurren sólo en un punto del país: han tenido lugar en las principales ciudades del territorio, Santiago, Valparaíso y Concepción, y en otras menores. Los autores no son delincuentes comunes: se interesan por antecedentes políticos de sus víctimas o por circunstancias similares a las que ellas aparecen vinculadas. [...] Los autores no carecen de medios para sus objetivos: poseen vehículos diversos, escondites seguros, medios de comunicación, información de actuaciones pasadas y actuales de las víctimas, sus familiares y otras relaciones''.\\
\begin{flushright}
\textit{Carta a Ricardo García, Ministro del Interior}, Vicaría de la Solidaridad. Santiago, 4 de julio de 1985.
\end{flushright}
El documento anterior se enmarca dentro del Caso Degollados, ocurrido en marzo de 1985, donde 3 profesionales del partido comunista fueron asesinados (entre ellos, un miembro de la Vicaría). Considerando la información entregada por la fuente, ¿a qué conclusión es posible llegar respecto a la violación de DD.HH. en dictadura?
\begin{flushleft}\begin{tabular}{@{\hspace{-.001\textwidth}}l@{\hspace{2pt}}p{.965\textwidth}}
A)& La dictadura persiguió a la Vicaría de la Solidaridad por su ideología de origen marxista.\\
B)& Las violaciones de DD.HH. eran realizadas por grupos ajenos al Estado, pero apoyados por este.\\
C)& La tortura era el único método utilizado por la dictadura para amedrentar a sus opositores.\\
D)& La CNI tenía alcance nacional y podía utilizar una enorme cantidad de recursos para causar terror.\\
E)& La carta fue enviada al ministro del interior, ya que era sabido que él dirigía las policías secretas.\\ 
\end{tabular}\end{flushleft}%correcta
\begin{key} D
\end{key} 
\begin{hint}
\end{hint}
\begin{answer} Correcta D \\
\end{answer}
\begin{info} %Información técnica de la pregunta
\begin{flushleft}
Eje Temático: HISTORIA: MUNDO, AMÉRICA Y CHILE.\\
Contenido: Las violaciones a los derechos humanos y la supresión del Estado de derecho durante la Dictadura Militar.\\
Habilidad: Análisis de fuente.\\
Dificultad: Difícil.\\
Capsula: Por determinar \\
\end{flushleft} 
\end{info}
\end{minipage}\vfill$\;$ %47

\begin{minipage}{\anchopregunta}
\item ``La franja se transmitió el lunes a las 22:45 de la noche y la derrota fue humillante: ``Los resultados fueron lamentables. Al cabo de muy pocos días nadie pudo ignorar la evidente superioridad técnica de la franja del No, mejor construcción argumental, mejores filmaciones, mejor música. Su melodía característica, en torno a la frase \textit{la alegría ya viene}, era tan pegajosa, que hasta los partidarios del Sí la tarareaban inconscientemente'', escribiría años después el entonces ministro del Interior, Sergio Fernández''.\\
\begin{flushright}
Sergio Fernández en la revista \textit{Capital}, 2012.
\end{flushright}
Durante la década de los 80's la televisión se fue masificando en los hogares chilenos debido, en parte, a la bonanza económica y al crédito. Tomando en cuenta el contexto sociocultural en el que se encuentra el Chile de 1988, ¿por qué fue relevante la franja electoral y la calidad de la Campaña del No?
\begin{flushleft}\begin{tabular}{@{\hspace{-.001\textwidth}}l@{\hspace{2pt}}p{.965\textwidth}}
A)& Debido a que el tiempo equitativo de la franja era el único momento que se tenía para difundir ideas.\\
B)& La efectividad de la campaña del No fue tal, que su victoria en el plebiscito fue de un 75\%.\\
C)& Ya que, al ser la televisión un medio masivo, una campaña política de calidad era fundamental.\\
D)& Como el No tenía mayor popularidad, las cadenas televisivas le dieron más tiempo al aire.\\
E)& Porque la franja electoral prohibía el uso de campañas del terror, beneficiando al bando del NO.\\ 
\end{tabular}\end{flushleft}%correcta
\begin{key} C
\end{key} 
\begin{hint}
\end{hint}
\begin{answer} Correcta C\\
\end{answer}
\begin{info} %Información técnica de la pregunta
\begin{flushleft}
Eje Temático: HISTORIA: MUNDO, AMÉRICA Y CHILE.\\
Contenido: Factores que incidieron en el proceso de recuperación de la democracia durante la década de 1980.\\
Habilidad: Pensamiento crítico.\\
Dificultad: Muy fácil, fácil.\\
Capsula: Por determinar \\
\end{flushleft} 
\end{info}
\end{minipage}\vfill$\;$ %48

\begin{minipage}{\anchopregunta}
\item El Frente Patriótico Manuel Rodríguez (FPMR) se fundó en 1983 con el fin de desestabilizar la dictadura militar y acabar con ella mediante la guerrilla urbana. Entre sus acciones más famosas podemos encontrar: secuestros, cortes de luz durante protestas, asaltos a camiones de mercancías y distribución de la misma y, finalmente, un fallido atentado contra Augusto Pinochet en 1986. Con el tiempo el FPMR fue perdiendo apoyo, pues el Partido Comunista decidió seguir la vía democrática y, posteriormente, la Unión Soviética se desintegró. De esta manera, es posible concluir que el FPMR, durante dictadura, fue:
\begin{flushleft}\begin{tabular}{@{\hspace{-.001\textwidth}}l@{\hspace{2pt}}p{.965\textwidth}}
A)& Un grupo que, mediante el terror, forzaron a la dictadura a llamar a plebiscito en 1988.\\
B)& Participe en los diálogos que se dieron entre los numerosos partidos de la oposición.\\
C)& Derrotado por las numerosas persecuciones de la CNI producto del atentado contra Pinochet.\\
D)& Exitoso en la desestabilización del gobierno militar al radicalizar a los sectores rurales.\\
E)& Relevante en sus acciones, pero el grupo tuvo poco impacto en el cambio político del país.\\ 
\end{tabular}\end{flushleft}%correcta
\begin{key} E
\end{key} 
\begin{hint}
\end{hint}
\begin{answer} Correcta E \\
\end{answer}
\begin{info} %Información técnica de la pregunta
\begin{flushleft}
Eje Temático: HISTORIA: MUNDO, AMÉRICA Y CHILE.\\
Contenido: Factores que incidieron en el proceso de recuperación de la democracia durante la década de 1980.\\
Habilidad: Pensamiento crítico.\\
Dificultad: Fácil, media.\\
Capsula: Por determinar \\
\end{flushleft} 
\end{info}
\end{minipage}\vfill$\;$ %49

\begin{minipage}{\anchopregunta}
\item Observe la siguiente imagen:
\begin{center}
    \includegraphics[width=5cm]{imagenes/imagen16.png}
\end{center}
La caída del muro de Berlín en noviembre de 1989 marca, para muchos, el fin simbólico de la Guerra Fría. La caída del muro fue pacífica, teniendo un festival espontáneo de abrazos y martillos. Con ello, la unificación y democratización de Alemania había comenzado, proceso que otras repúblicas soviéticas seguirán. Considerando el contexto entregado y sus conocimientos previos, ¿en qué se parece la caída del muro de Berlín y la desintegración de los satélites soviéticos?
\begin{flushleft}\begin{tabular}{@{\hspace{-.001\textwidth}}l@{\hspace{2pt}}p{.965\textwidth}}
A)& Exceptuando Rumania, fueron procesos pacíficos y sin intervención de la URSS.\\
B)& Fueron gatillados por agentes de inteligencia occidentales para destruir el modelo soviético.\\
C)& Fueron desencadenados por la caída de Unión Soviética tras el golpe de Boris Yeltsin.\\
D)& Implicaron grandes transformaciones geográficas debido a la unificación de territorios.\\
E)& El descontento económico por la Glasnot fue la causa que motivó las protestas violentas iniciales.\\ 
\end{tabular}\end{flushleft}%correcta
\begin{key} A
\end{key} 
\begin{hint}
\end{hint}
\begin{answer} Correcta A \\
\end{answer}
\begin{info} %Información técnica de la pregunta
\begin{flushleft}
Eje Temático: HISTORIA: MUNDO, AMÉRICA Y CHILE\\
Contenido: El mundo después de la Guerra Fría\\
Habilidad: Pensamiento temporal y espacial.\\
Dificultad: Difícil, muy difícil.\\
Capsula: Por determinar \\
\end{flushleft} 
\end{info}
\end{minipage}\vfill$\;$ %60

\begin{minipage}{\anchopregunta}
\item En 1992 Francis Fukuyama publica su polémico libro ``El Fin de la Historia'' que va a servir de base para el pensamiento neoconservador en Occidente. La tesis de su libro se puede resumir en que, tras el fin de la Guerra Fría y la derrota de la Unión Soviética, las democracias liberales y capitalistas habían triunfado como modelo político. La tesis de Fukuyama se ha vuelto controversial en los años posteriores, pues el mundo post Guerra Fría no ha sido uno de absoluta hegemonía occidental. Un fenómeno histórico que sirve de contrapeso a la teoría de hegemonía de modelo político de Fukuyama la podemos encontrar en: 
\begin{flushleft}\begin{tabular}{@{\hspace{-.001\textwidth}}l@{\hspace{2pt}}p{.965\textwidth}}
A)& La aparición de China como potencia mundial, representante de un comunismo tradicional.\\
B)& La inestabilidad política en el antes pacífico Tercer Mundo tras la caída de la URSS.\\
C)& Las diferencias político-económicas entre Europa y EEUU que han debilitado a la OTAN.\\
D)& El ascenso de fanáticos religiosos en el Medio Oriente y el yihadismo internacional tras el 9-11.\\
E)& La aplicación del desarrollo sustentable, que rompe completamente con el capitalismo.\\ 
\end{tabular}\end{flushleft}%correcta
\begin{key} D
\end{key} 
\begin{hint}
\end{hint}
\begin{answer} Correcta D \\
\end{answer}
\begin{info} %Información técnica de la pregunta
\begin{flushleft}
Eje Temático: HISTORIA: MUNDO, AMÉRICA Y CHILE\\
Contenido: El mundo después de la Guerra Fría\\
Habilidad: Pensamiento temporal y espacial.\\
Dificultad: Difícil, muy difícil.\\
Capsula: Por determinar \\
\end{flushleft} 
\end{info}
\end{minipage}\vfill$\;$ %59

%%%%% ECONOMÍA Y SOCIEDAD %%%%%


\begin{minipage}{\anchopregunta}
\item El problema económico se puede entender como el balance entre recursos limitados (escasez) y necesidades infinitas. Pero las necesidades, además, son progresivas. Considerando ambas características de las necesidades, ¿de qué forma se pueden relacionar con las sociedades durante la historia?
\begin{flushleft}\begin{tabular}{@{\hspace{-.001\textwidth}}l@{\hspace{2pt}}p{.965\textwidth}}
A)& Los sistemas productivos y sus cambios producen bajas en la producción, trayendo necesidad.\\
B)& Se relacionan mediante la idea de consumismo, creando necesidades artificiales contemporáneas.\\
C)& Las necesidades tienen un límite con las sociedades industriales, pero son ilimitadas por la demografía. \\
D)& A medida que las sociedades avanzan, se cubren ciertas necesidades, y aparecen otras más complejas.\\
E)& Las sociedades que se extienden a lo largo del tiempo tienden a crear pobres y necesitados.\\
\end{tabular}\end{flushleft}%correcta
\begin{key} D
\end{key} 
\begin{hint}
\end{hint}
\begin{answer} Correcta D \\
\end{answer}
\begin{info} %Información técnica de la pregunta
\begin{flushleft}
Eje Temático: ECONOMÍA Y SOCIEDAD.\\
Contenido: El problema económico de la escasez y las necesidades ilimitadas y relaciones económicas de los distintos agentes.\\
Habilidad: Pensamiento crítico.\\
Dificultad: Difícil, muy difícil.\\
Capsula: Por determinar \\
\end{flushleft} 
\end{info}
\end{minipage}\vfill$\;$ %51

\begin{minipage}{\anchopregunta}
\item Existen dos clasificaciones de fuentes energéticas: No renovables (combustibles fósiles, minerales) y renovables (energía eólica, solar, geotérmica, etc.). Considerando que la energía es una de las mayores necesidades de la humanidad actual, ¿por qué la existencia de las energías renovables no anula el principio de recursos limitados?
\begin{flushleft}\begin{tabular}{@{\hspace{-.001\textwidth}}l@{\hspace{2pt}}p{.965\textwidth}}
A)& Porque, con el tiempo, incluso las energías no renovables pueden regenerarse.\\
B)& Porque los recursos se consumen más rápido de los que pueden recuperarse.\\
C)& Porque las energías renovables dependen de variables muy complejas de predecir.\\
D)& Porque las energías renovables son algo reciente que todavía no demuestra su calidad.\\
\end{tabular}\end{flushleft}%correcta
\begin{key} B
\end{key} 
\begin{hint}
\end{hint}
\begin{answer} Correcta B \\
\end{answer}
\begin{info} %Información técnica de la pregunta
\begin{flushleft}
Eje Temático: ECONOMÍA Y SOCIEDAD.\\
Contenido: El problema económico de la escasez y las necesidades ilimitadas y relaciones económicas de los distintos agentes.\\
Habilidad: Pensamiento crítico.\\
Dificultad: Difícil, muy difícil.\\
Capsula: Por determinar \\
\end{flushleft} 
\end{info}
\end{minipage}\vfill$\;$ %52

\begin{minipage}{\anchopregunta}
\item ``Para evitarlo, De Beers lanzó una campaña que podría ser considerada una de las grandes obras maestras del marketing. ``Un diamante es para siempre'', era el eslogan. Con ello transmitían la idea de que, como el amor, el diamante se quedaba para siempre con una. Lograron, de hecho, calar en el subconsciente ese valor emocional de la piedra y que no se vendiese en caso de necesidad (tanto que incluso ahora el mercado de segunda mano de diamantes es muy bajo).\\
Las campañas que unían diamantes y amor y su permanencia por siempre fueron tan eficientes que la compañía los llevó a otros mercados décadas más tarde. En Japón lograron que en unas décadas el 70\% de las novias recibiesen un anillo de diamantes en su petición de manos. Cuando empezaron con su estrategia de marketing, partían de una situación de prácticamente cero''.
\begin{flushright}
\textit{Cómo los diamantes se convirtieron en la piedra del amor y la estrategia de marketing que los encumbró}, Puromarketing, 2020.
\end{flushright}
La empresa De Beers en un momento tuvo el monopolio sobre la producción de diamantes, siendo uno de los factores del valor de los diamantes. Posteriormente, perderían el monopolio, pero uno de los factores que alteran el precio de los bienes se mantendría. Según la fuente, ¿cuál de los siguientes elementos puede generar una subida de precios de un producto?
\begin{flushleft}\begin{tabular}{@{\hspace{-.001\textwidth}}l@{\hspace{2pt}}p{.965\textwidth}}
A)& El valor agregado de un producto al ser trabajado por expertos.\\
B)& La aparición de nuevas tecnologías que requieran materia prima.\\
C)& La ampliación de una empresa productora hacia mercados internacionales.\\
D)& El uso de campañas publicitarias que apelen a sentimentalismo.\\
E)& El mercado competitivo entre empresas productoras de un bien.\\ 
\end{tabular}\end{flushleft}%correcta
\begin{key} D
\end{key} 
\begin{hint}
\end{hint}
\begin{answer} Correcta D\\
\end{answer}
\begin{info} %Información técnica de la pregunta
\begin{flushleft}
Eje Temático: ECONOMÍA Y SOCIEDAD\\
Contenido: El funcionamiento del mercado y los factores que pueden alterarlo.\\
Habilidad: Análisis de fuentes.\\
Dificultad: Fácil, media.\\
Capsula: Por determinar \\
\end{flushleft} 
\end{info}
\end{minipage}\vfill$\;$ %53

\begin{minipage}{\anchopregunta}
\item ``Un bulbo virtual podía cambiar hasta 10 veces de manos en un solo día, generando beneficios en cada uno de los intercambios.  El precio subió entre un 500 y un 2000\% en tan solo unas semanas. La gente invertía todo lo que tenía en comprar al menos uno de aquellos pagarés.
Muchos se endeudaron de por vida para poder participar holgadamente en el mercado del tulipán. Hipotecaban su casa, sus tierras, sus pertenencias o, incluso, prometiendo décadas de sus futuros sueldos.  Ya no se trataba de la vanidad de adornar el jardín con las mejores flores del continente. La importancia del tulipán iba mucho más allá. Se había construido de la nada un mercado financiero de futuros en el que estaba involucrada toda la sociedad.''\\
\begin{flushright}
\textit{La tulipomanía, la primera burbuja económica de la historia}, Pepe Pinel, BBVA.
\end{flushright}
La tulipomanía se dio durante el siglo XVII en los Países Bajos. En un inicio el mercado de esta burbuja se daba mediante el intercambio de bulbos de forma física, aunque el hecho de que fueran efímeros era una limitante. Para ello, se recurrió a lo descrito por la fuente, llevando al colapso de dicho mercado. Considerando lo planteado por la fuente y sus conocimientos sobre el mercado, ¿qué cambio terminó causando el derrumbe de la tulipomanía?
\begin{flushleft}\begin{tabular}{@{\hspace{-.001\textwidth}}l@{\hspace{2pt}}p{.965\textwidth}}
A)& La burbuja de la clase alta llegó a un final cuando estos no pudieron seguir comprando.\\
B)& Los créditos masivos, como en todo mercado a futuro, dejaron a los bancos en la quiebra.\\
C)& Al masificarse el mercado, los compradores se dieron cuenta de la inutilidad de este.\\
D)& El ingreso desmedido de capital endeudado ocasionó la excesiva especulación financiera.\\
E)& La falta de conocimiento sobre botánica conllevó la aparición de virus en los bulbos.\\ 
\end{tabular}\end{flushleft}%correcta
\begin{key} D
\end{key} 
\begin{hint}
\end{hint}
\begin{answer} Correcta D \\
\end{answer}
\begin{info} %Información técnica de la pregunta
\begin{flushleft}
Eje Temático: ECONOMÍA Y SOCIEDAD\\
Contenido: El funcionamiento del mercado y los factores que pueden alterarlo.\\
Habilidad: Análisis de fuentes.\\
Dificultad: Media.\\
Capsula: Por determinar \\
\end{flushleft} 
\end{info}
\end{minipage}\vfill$\;$ %54

\begin{minipage}{\anchopregunta}
\item Observe la siguiente imagen:
\begin{center}
    \includegraphics[width=9cm]{imagenes/imagen15.png}
\end{center}
\begin{flushright}
\textit{La Nación de Argentina}, GDA.
\end{flushright}
El corralito fue una medida que se adoptó en Argentina en 2001, donde se congelaban los capitales de los ciudadanos en los bancos con el objetivo de evitar la fuga de capital. Los argentinos, como el hombre en la imagen, no estuvieron contentos con la medida. Considerando la medida tomada por el gobierno argentino y el descontento de su población, ¿qué función económica de los bancos se ve restringida con el corralito?
\begin{flushleft}\begin{tabular}{@{\hspace{-.001\textwidth}}l@{\hspace{2pt}}p{.965\textwidth}}
A)& El permitir el flujo de capital mediante el retiro de ahorros.\\
B)& El fomento de mercado mediante el préstamo de dinero a pymes.\\
C)& Controlar el circulante mediante la acuñación de monedas y billetes.\\
D)& El robo de dinero al usarlo de forma indebida en proyectos estatales.\\
E)& La asesoría económica hacia las personas que deseen empezar a ahorrar.\\ 
\end{tabular}\end{flushleft}%correcta
\begin{key} A
\end{key} 
\begin{hint}
\end{hint}
\begin{answer} Correcta A \\
\end{answer}
\begin{info} %Información técnica de la pregunta
\begin{flushleft}
Eje Temático: ECONOMÍA Y SOCIEDAD\\
Contenido: Características de algunos instrumentos financieros, considerando los riesgos y beneficios derivados de su uso.\\
Habilidad: Pensamiento crítico.\\
Dificultad: Muy fácil, fácil.\\
Capsula: Por determinar \\
\end{flushleft} 
\end{info}
\end{minipage}\vfill$\;$ %54

\begin{minipage}{\anchopregunta}
\item Los bancos pueden ser considerados como agentes económicos, por lo que tienen un rol de suma importancia en el mercado. La función principal de los bancos es almacenar dinero que no está en uso (ahorros) para generar más ganancias, además de dar prestamos a otras personas y ciudadanos. Considerando los servicios que ofrece un banco, ¿de qué manera mantienen su rentabilidad?
\begin{flushleft}\begin{tabular}{@{\hspace{-.001\textwidth}}l@{\hspace{2pt}}p{.965\textwidth}}
A)& Creando cuentas de ahorro con tasas de interés bajas.\\
B)& Mediante los intereses cobrados a los prestamistas.\\
C)& Poseyendo acciones en las empresas que piden préstamos.\\
D)& Al permitir que los ahorros generen dinero por sí solos.\\
E)& Utilizando bonos de seguridad entregados por el Estado. \\ 
\end{tabular}\end{flushleft}%correcta
\begin{key} B
\end{key} 
\begin{hint}
\end{hint}
\begin{answer} Correcta B\\
\end{answer}
\begin{info} %Información técnica de la pregunta
\begin{flushleft}
Eje Temático: ECONOMÍA Y SOCIEDAD\\
Contenido: Características de algunos instrumentos financieros, considerando los riesgos y beneficios derivados de su uso.\\
Habilidad: Pensamiento crítico.\\
Dificultad: Media.\\
Capsula: Por determinar \\
\end{flushleft} 
\end{info}
\end{minipage}\vfill$\;$ %55

\begin{minipage}{\anchopregunta}
\item Un consumidor informado tiene un problema con su banco debido a cobros indebidos al momento de hacer movimientos en su cuenta de ahorros que, además, no genera los intereses planteados en el contrato. Considerando la situación en la que se encuentra el consumidor, ¿a qué organismo especializado debería contactar para que fiscalice y regule el accionar de su banco?
\begin{flushleft}\begin{tabular}{@{\hspace{-.001\textwidth}}l@{\hspace{2pt}}p{.965\textwidth}}
A)& Fiscalía Nacional Económica (FNE)\\
B)& Superintendencia de Bancos e Instituciones Financieras (SBIF)\\
C)& Comisión de Mercado Financiero (CMF)\\
D)& Superintendencia de Pensiones (SP)\\
E)& Servicio Nacional del Consumidor (SERNAC)\\ 
\end{tabular}\end{flushleft}%correcta
\begin{key} C
\end{key} 
\begin{hint}
\end{hint}
\begin{answer} Correcta C\\
\end{answer}
\begin{info} %Información técnica de la pregunta
\begin{flushleft}
Eje Temático: ECONOMÍA Y SOCIEDAD\\
Contenido: Evaluación de situaciones de consumo informado y responsable considerando derechos y deberes del consumidor, sentido del ahorro y del endeudamiento, entre otros.\\
Habilidad: Pensamiento Crítico.\\
Dificultad: Fácil, media.\\
Capsula: Por determinar \\
\end{flushleft} 
\end{info}
\end{minipage}\vfill$\;$ %56

\begin{minipage}{\anchopregunta}
\item ``Tras las tres jornadas del evento Cyber Monday, el SERNAC recibió 488 reclamos de parte de las y los consumidores, quienes se quejaron principalmente por: incumplimientos de las promociones y ofertas, publicidad engañosa, anulaciones unilaterales de compras, falta de stock, no respetar los precios al momento de realizar el pago, entre otros.
Por otro lado, el SERNAC, de acuerdo al análisis preliminar de los hallazgos detectados durante la fiscalización realizada durante los tres días del evento, constató, por ejemplo, que algunas empresas no informan adecuadamente del derecho de garantía legal y cómo ejercerlo, además de establecer ciertas restricciones para su ejercicio. Además, se están analizando otros antecedentes recopilados para determinar si hubo ofertas engañosas, entre otros incumplimientos a la norma.''\\
\begin{flushright}
\textit{Cerca de 500 reclamos recibió el SERNAC durante el Cyber Monday}, SERNAC, 05/10/2023.
\end{flushright}
De acuerdo a la información entregada por la fuente, es correcto afirmar que para un correcto funcionamiento del SERNAC:
\begin{flushleft}\begin{tabular}{@{\hspace{-.001\textwidth}}l@{\hspace{2pt}}p{.965\textwidth}}
A)& Estar atento solo a eventos con alta circulación de bienes y servicios.\\
B)& El SERNAC debe actuar por cuenta propia, ya que los consumidores exageran.\\
C)& Debe existir una legislación más robusta sobre los derechos del consumidor.\\
D)& Debe existir un gobierno afín que potencie las capacidades del SERNAC.\\
E)& Los consumidores deben informar cuando sus derechos sean vulnerados.\\ 
\end{tabular}\end{flushleft}%correcta
\begin{key} E
\end{key} 
\begin{hint}
\end{hint}
\begin{answer} Correcta E \\
\end{answer}
\begin{info} %Información técnica de la pregunta
\begin{flushleft}
Eje Temático: ECONOMÍA Y SOCIEDAD\\
Contenido: Evaluación de situaciones de consumo informado y responsable considerando derechos y deberes del consumidor, sentido del ahorro y del endeudamiento, entre otros.\\
Habilidad: Análisis de fuente.\\
Dificultad: Fácil, media.\\
Capsula: Por determinar \\
\end{flushleft} 
\end{info}
\end{minipage}\vfill$\;$ %57

% Muy fácil, Fácil, Media, Díficil, muy díficil.
\end{enumerate}
\end{document}